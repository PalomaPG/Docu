\chapter{Conclusi\'on}
\label{ch:conclusion}

Se propuso una nueva versi\'on optimizada del software implementado originalmente por Pablo Huentelemu destinado a la detecci\'on de supernovas, obteni\'endose tambi\'en una familia de filtros de Kalman para el an\'alisis de im\'agenes astron\'omicas que puede ser f\'acilmente extendida con la finalidad de agregar nuevos m\'etodos de estimaci\'on de estados. Esta extensi\'on es facilitada gracias al patr\'on de dise\~no Strategy empleado en el dise\~no.
\bigskip

Por otro lado, otro de los beneficios adquiridos en esta nueva versi\'on es el control sobre los argumentos de entrada del programa. Anteriormente, las expresiones regulares de los nombres de los archivos ten\'ian que ser escritos expl\'icitamente en el c\'odigo del programa. Sin embargo el c\'odigo redise\~nado en este trabajo permite la recepci\'on de estas expresiones regulares en un archivo de texto plano por lo que el usuario s\'olo debe entregar las rutas y expresiones regulares bien estructuradas de las im\'agenes en este documento. Adem\'as, cabe destacar que igualmente se independizaron del c\'odigo los valores de entrada de umbrales y de par\'ametros requeridos en las diferentes rutinas entreg\'andose estos en un  archivo de texto igualmente en donde se incluye el campo y su valor. Es decir se logr\'o evitar el \textit{hardcoding} presente en la versi\'on anterior de la pipeline.
\bigskip

Se debe agregar, que dentro de los cambios agregados a la nueva versi\'on de la pipeline se acort\'o tiempo del proceso ya que a priori se asume que no se conoce ninguna supernova, es decir, el programa no tiene porqu\'e saber si existe un candidato conocido, por lo que los resultados de los potenciales aspirantes a supernova son tratados ecu\'animamente, y por tanto no se discrimina en la informaci\'on obtenida en los resultados al ser guardada en disco. Por ende, el an\'alisis se realiza en una pasada (lo que le permite demorarse la mitad del tiempo original) y no en dos como se realizaba antiguamente.
\bigskip

Respecto de las pruebas realizadas en la evaluaci\'on de las versiones original y nueva, se observa que efectivamente la nueva versi\'on de los filtros b\'asico y de m\'axima correntrop\'ia logran detectar m\'as supernovas que sus respectivas versiones originales (considerando el n\'umero de datasets sobre los cuales se trabaj\'o), adem\'as de reducir el n\'umero de falsos positivos para el mismo conjunto de umbrales. Este resultado se presume que puede ser debido a la mejora en el ordenamiento de los datos de entrada ya que este ordenamiento es aplicado sobre todos los datasets de entrada, tanto a im\'agenes FITS como a archivos de extensi\'on NPY necesarios para la rutina y no s\'olo sobre las primeras como lo realiza el programa original. Del an\'alisis de desempe\~no se desprende que entre las versiones antigua y refactorizada se observa que hay una mejora los tiempos de procesamiento. Sin embargo se mantiene el uso de memoria principal, tanto para el filtro b\'asico como el de m\'axima correntrop\'ia. 
\bigskip

Se refactoriz\'o igualmente el proceso de visualizaci\'on de resultados; con esta nueva versi\'on no s\'olo es posible observar las series de tiempo de la evoluci\'on de las mediciones y observar el comportamiento de los p\'ixeles, flujos y estimaciones gr\'aficamente (como estampillas) sino adem\'as se puede visualizar el resultado de las estimaciones de estado para alg\'un candidato espec\'ifico en el espacio de fase, indicando fecha de detecci\'on (como MJD) y el valor de la complejidad de la curva generada en t\'erminos de su entrop\'ia.
\bigskip

Se implement\'o adem\'as un nuevo miembro de la familia de filtros de Kalman el cu\'al permite realizar aproximaciones con funciones que dependan del paso del tiempo respecto de alg\'un instante de referencia particular. Sin embargo los primeros resultados de este filtro dan cuenta de que se hace necesario estudiar y probar nuevos conjuntos de par\'ametros, condiciones iniciales y funciones, ya que para este trabajo s\'olo se utiliz\'o un rango acotado de estos elementos. Igualmente falt\'o explorar el comportamiento del filtro de m\'axima correntrop\'ia usando el m\'etodo de Silverman.  
\bigskip

Por otra parte se estima que el hecho de que los datos usados para la realizaci\'on de este trabajo presenten un muestreo irregular complica la obtenci\'on de buenos resultados en el proceso de detecci\'on ya que el proceso de correcci\'on requiere las observaciones durante la estimaci\'on de estados con el filtro de Kalman.
\bigskip 

\section{Trabajo futuro}

Uno de los aspectos pendientes y que se esperar\'ia trabajar en el futuro es el estudiar la variaci\'on de resultados del filtro unscented usando diferentes funciones no lineales. Adem\'as investigar que sucede al variar los par\'ametros como $\sigma_a$. De la misma forma ser\'ia deseable explorar lo que ocurre con el filtro de m\'axima correntrop\'ia al momento de usar el m\'etodo de Silverman.
\bigskip

Por otra parte queda tambi\'en pendiente, el estudiar el comportamiento de los resultados al variar los umbrales relacionados con la estimaci\'on realizada por el filtro de Kalman y que son usados en la fase de reconocimiento de candidatos (en la clase \textsc{SourceFinder}).
\bigskip

Los m\'etodos presentados podr\'ian ser aplicados a la detecci\'on de estrellas variables aplicando un filtro que distinga tendencias decrecientes en la luminosidad de estos objetos; por lo que una extensi\'on prometedora de este sistema podr\'ia dise\~narse  para reconocer alternancia en los reg\'imenes creciente y decreciente, haciendo uso de la estructura de clases dada por el patr\'on Strategy en el modelo del filtro.
\bigskip


