\chapter{Conclusiones}
\label{ch:conclusion}



\section{Trabajo futuro}
El trabajo propuesto en esta tesis describe tres diferentes versiones de un nuevo m\'etodo de detecci\'on de fen\'omenos trasientes en r\'egimen creciente, basados tres variantes del filtro de Kalman, destinado a la detecci\'on de supernovas en etapa temprana. Sin embargo este mismo m\'etodo puede ser aplicado en la detecci\'on de estrellas variables usado un filtro que distinga tendencias decrecientes en la luminosidad de estos objetos; por lo que una extensi\'on prometedora de este sistema podr\'ia dise\~narse  para reconocer alternancia en los reg\'imenes creciente y decreciente, haciendo uso de la estructura de clases dada por el patr\'on Strategy en el modelo del filtro.
\bigskip

Por otro lado queda tambi\'en pendiente, el estudiar el comportamiento de los resultados al variar los umbrales relacionados con la estimaci\'on realizada por el filtro de Kalman y que son usados en la fase de reconocimiento de candidatos (en la clase \textsc{SourceFinder}). [M\'etodo basado en machine learning para estudiar falsos positivos??] 

