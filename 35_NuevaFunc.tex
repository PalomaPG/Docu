\chapter{Filtro de aproximaci\'on no lineal: unscented Kalman}
\label{ch:news}
En este cap\'itulo se describe la estructura y detalles importantes del nuevo miembro de la familia de clases de filtros de Kalman: \textsc{UnscentedKalman}


\section{\textsc{Dise\~no}}
El dise\~no de la nueva clase de filtro se realiz\'o continuando con la arquitectura del patr\'on Strategy, definiendo los m\'etodos de predicci\'on y correcci\'on en instancias de clases que implementen las interfaces \textsc{IPredict} e \textsc{ICorrect} respectivamente: \textsc{UnscentedPredict} y \textsc{UnscentedCorrect}.
\bigskip

El filtro unscented se desarroll\'o en la clase \textsc{UnscentedKalman}, la cual encierra los m\'etodos de predicci\'on y correcci\'on (descritos en el Cap\'itulo \ref{ch:background}, subsecci\'on \ref{ssec:ukf}) implementados en \texttt{predict} y \texttt{correct}.
\bigskip

El constructor de esta clase requiere como argumentos de entrada los siguientes par\'ametros:

\begin{itemize}
\item \texttt{f\_func:} Funci\'on usada para obtener el primer conjunto de puntos sigma para el per\'iodo de predicci\'on. Consiste en una funci\'on vectorial que se aplica sobre cada elemento de una matriz cuya dimensi\'on es la misma que la de la matriz de estado. Corresponde a la funci\'on principal con la que se aplicar\'a el modelo no lineal en el proceso de predicc\'on.
\item \texttt{h\_func:} Funci\'on usada para obtener el segundo conjunto de puntos sigma durante el proceso de correcci\'on. Su aplicaci\'on procede de la misma forma que la descrita para \texttt{f\_func}. Participa en el proceso de correcci\'on.
\item \texttt{f\_args:} Corresponde a una lista de n\'umeros reales de largo dos. El primer valor corresponde a la potencia o grado de la funci\'on mientras que el segundo t\'ermino corresponde al factor que la multiplica. Estos valores est\'an asociados a la funci\'on \texttt{f}.
\item \texttt{h\_args:} Corresponde a una lista de n\'umero reales de largo dos. El primer valor corresponde a la potencia o grado de la funci\'on mientras que el segundo t\'ermino corresponde al factor que la multiplica. Estos valores est\'an asociados a la funci\'on \texttt{h}.
\item \texttt{alpha:} T\'ermino que indica que tan separados se encuentran los puntos sigma en torno a la media (junto a \texttt{kappa}). Por lo general su valor oscila entre $10^{-4}$ y $1$. 
\item \texttt{beta:} Describe el valor de $\beta$ el cual incorpora conocimiento \textit{a priori} sobre la distribuci\'on de las variables de estado. Para distribuciones gaussianas, tiene un valor de $\beta=2$. 
\item \texttt{kappa:} Corresponde a un par\'ametro de escalamiento secundario en la distancia de los puntos sigma a la media. Su valor va entre $0$ y $3-N$ \cite{wan}.
\item \texttt{sigma\_a:} Varianza asociada a la variable de control $\Delta t$. La funci\'on no lineal se aplicar\'a sobre el paso del tiempo.
\item \texttt{image\_size:} Dimensiones de la imagen de flujo. Corresponde a una tupla de dos n\'umeros enteros (por defecto, $(4094, 2046)$).
\end{itemize}

Al inicializar una instancia de esta clase, se calculan tanto los pesos como el t\'ermino $\lambda$, los cuales dependen de los par\'ametros: $\beta$, $\kappa$ $\alpha$ y $N$ (o cantidad de variables de estado). Ver Ecuaciones \ref{eq:eq20} y \ref{eq:eq21}.
\bigskip

Como parte de la familia de \textsc{KalmanFilter} posee el m\'etodo \texttt{update}, desde el cual se llama a \texttt{predict} y a \texttt{correct}, conservando la firma. 
 
\subsection{Predicci\'on}
La predicci\'on de este filtro se implement\'o, como se mencion\'o anteriormente, en la clase \textsc{UnscentedPredict} (que implementa a \textsc{IPredict}). Para instanciar esta clase se debe entregar como argumento al constructor: un puntero a funci\'on (que puede o no ser lineal) \texttt{f\_func} y sus argumentos en \texttt{f\_args}; los pesos asociados a la media, \texttt{W\_m}, y a la covarianza, \texttt{W\_c}; el coeficiente $\lambda$ (\texttt{lambda\_}); la cantidad de variables de estado medidas, $N$ (en este modelo, se trata de dos: flujo y velocidad de flujo, por ende $N=2$); la varianza $\sigma_a$ (\texttt{sigma\_a}) y la dimensi\'on de las im\'agenes \texttt{image\_size}. Por lo tanto la firma del constructor queda como sigue:
\bigskip
\begin{center}
\texttt{UnscentedPredict(f\_func, f\_args, Wm, Wc, lambda\_, N, sigma\_a, image\_size)}
\end{center}
\bigskip

Se destaca que tantos los pesos como el valor de $\lambda$ se mantienen constantes durante todo el proceso de predicci\'on y estimaci\'on, valores con los cuales se obtienen los puntos sigma y el valor medio al propagar la funci\'on \texttt{f\_func} (ver Ecuaci\'on \ref{eq:eq18}).
\bigskip

La salida de este m\'etodo est\'a compuesta por la matriz de estado predicha, la predicci\'on de la covarianza y los puntos sigma guardados en la variable \texttt{Xs} de la implementaci\'on.

\subsection{Correcci\'on}
El proceso de correcci\'on est\'a implementado en la clase \textsc{UnscentedCorrect} la cual implementa la interfaz \textsc{ICorrect} redefiniendo el m\'etodo \texttt{correct} para esta versi\'on del filtro. De acuerdo al proceso de correcci\'on para el funcionamiento del filtro se deben definir las funciones $f(\cdot)$ y $h(\cdot)$, adem\'as requiere de los pesos calculados previmente (Ecuaci\'on \ref{eq:eq20}).
\bigskip

El constructor de esta clase posee la siguiente firma.

\bigskip
\begin{center}
\texttt{UnscentedCorrect(f\_func, h\_func, f\_args, h\_args,  Wm, Wc, lambda\_, N, image\_size)}
\end{center}

Durante la correcci\'on se escogen nuevamente $2N+1$ puntos sigma en torno al estado predicho (fase previa) (ver Ecuaci\'on \ref{eq:eq24}), sin embargo los valores de los pesos y de $\lambda$ son los mismos usados durante la fase de predicc\'on, cambiando s\'olo la funci\'on que se propaga sobre este nuevo conjunto de puntos: $h(\cdot)$ (expresada como \texttt{h\_func}). Las operaciones que participan en este proceso se implementaron en el script \texttt{unscented\_utils} (para ver detalles de los m\'etodos, ver Ap\'endice, secci\'on \ref{ap:unsutils}).
\bigskip

La familia de filtros resultantes de la agregaci\'on de este nuevo m\'odulo se puede observar en la Figura \ref{fig:kalmanfilter2}.

\begin{figure}
\centering
\includegraphics[scale=.5]{/home/paloma/Documents/Memoria/Code/sif2/kalmanfilter2.png}
\caption{Diagrama de clases de la familia de filtros de Kalman resultante. El patr\'on Strategy permite que las funcionalidades de predicci\'on y correcci\'on puedan definirse en diferentes clases, facilitanto la modularizaci\'on y la implementaci\'on de nuevos filtros como el unscented. }
\label{fig:kalmanfilter2}
\end{figure}