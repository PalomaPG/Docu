\documentclass[letterpaper]{templates/uchile-tesis}
% Comandos especiales de esta tesis
%\newcommand{\mb}{\textit{microblogging}\xspace}
%\newcommand{\web}{Web\xspace}
%\newcommand{\tw}{\textit{Twitter}\xspace}
%\newcommand{\etal}{\textit{et. al.}\xspace}

% Portada - Variables
\facultad{Facultad de Ciencias Físicas y Matemáticas} 
\departamento{Departamento de Ciencias de la Computación}
\title{Extensi\'on de filtro de Kalman de aproximaci\'on no lineal para la detecci\'on de objetos astronómicos}

\trabajoygrado{Tesis para optar al título de Ingeniera Civil en Computación}
%\trabajoysubgrado{Memoria para optar al título de Ingeniero Civil Industrial}


\author{Paloma Pérez Garc\'ia}

\profguia{Sr. Pablo Estevez Valencia} %profesor guia
\profcoguia{Sr. Benjam\'in Bustos} %profesor co-guia
\profint{Sr. ZZZ ZZZ ZZZ} %profesor integrante
\profinta{Sr. ZZ ZZ ZZ} %profesor integrante 2, generalemente no es necesario
\profintb{Sr. ZZZ ZZZ ZZZ} %profesor integrante 3, generalmente no es necesario
%\proyecto{Financiado por el proyecto \#ZZZZ}


\ciudad{Santiago} \pais{Chile} \monthpub{Diciembre} \yearpub{2018}

\begin{document}

% Lista de TODOS y FIXMES, no aparece si es que no hay nada que hacer
%\listoftodos
\newpage

%% Portada
\maketitle

% Prefacio


%\hfill
%\begin{tabular}{@{}l@{}}
%\uppercase{Resumen de la memoria para optar al}\\
%\uppercase{t\'itulo de ingeniera civil en computaci\'on}\\
%\uppercase{Por: Paloma Cecilia P\'erez Garc\'ia}\\
%\uppercase{Fecha: 2019}\\
%\uppercase{Prof. Gu\'ia: Pablo Est\'evez Valencia}\\
%\uppercase{Prof. Co-Gu\'ia: Benjam\'in Bustos C\'ardenas}\\
%\end{tabular}
%\bigskip

\begin{preface}
\begin{center}
%\uppercase{Extensi\'on de filtro de Kalman de aproximaci\'on no lineal para la detecci\'on de objetos astronómicos}
\uppercase{Resumen}
\end{center}
%\section{Resumen Ejecutivo}
El presente trabajo describe el desarrollo de un software en \textsc{Python} destinado a la detecci\'on de fen\'omenos astron\'omicos transitorios como las supernovas que corresponden a eventos caracterizados por un incremento r\'apido en su luminosidad y un consecuente decremento lento en esta. El programa se dise\~n\'o sobre la base de una rutina ya implementada la cual hace uso de estimaciones generadas por m\'etodos del filtro de Kalman. En su versi\'on cl\'asica (o b\'asica) y tambi\'en de m\'axima correntrop\'ia. Debido a que esta rutina presenta complicaciones en la administraci\'on de archivos y manejo de par\'ametros (producido principalmente por \textit{hard-coding}) se realiz\'o un proceso de \textit{refactoring} que implica adem\'as dise\~nar y generar una nueva familia de filtros de Kalman basados en el patr\'on de dise\~no \textsc{Strategy}.
\bigskip

Sobre este c\'odigo refactorizado se efectuaron pruebas de rendimiento obteni\'endose as\'i una mejora en t\'erminos de tiempo pero no en la memoria principal utilizada. Por otro lado se realizaron pruebas de detecci\'on usando el conjunto de 93 supernovas detectadas por el sondeo de HiTS del a\~no 2015, hall\'andose mejoras notables en la disminuci\'on de falsos positivos as\'i como tambi\'en un leve aumento en el n\'umero de verdaderos positivos al emplear las versiones cl\'asica y de m\'axima correntrop\'ia de los filtros refactorizados. Sin embargo no ocurri\'o lo mismo con el filtro unscented, que permite emplear funciones no lineales al momento de estimar. Para este filtro se usaron una funci\'on cuadr\'atica y otra de exponente 1.5, evaluadas sobre el paso del tiempo desde el inicio de las observaciones (o \'epocas).   
\bigskip

Se recomienda continuar estudiando el nuevo filtro de Kalman de aproximaci\'on no lineal debido al acotado conjunto de par\'ametros y funciones utilizado durante la realizaci\'on de este trabajo.


% Pagina Optativa - Dedicatoria
\dedicatoria{En memoria de H. P.}

% Pagina Optativa - Agradecimientos
\section{Agradecimientos}

\begin{flushright}
\makeatletter
	\@author
\makeatother
\end{flushright}

% Indice - General
\tableofcontents

% Indice - Tablas
\listoftables

% Indice - Figuras
\listoffigures

\end{preface}

% Introducción
\chapter{Introducción}
\label{ch:introduction}

\section{Motivaci\'on}
La astronomía es uno de los campos científicos que más se ha visto afectado por el rápido crecimiento en la generación de datos debido al fuerte desarrollo de nuevas tecnologías de la información y  de nuevos instrumentos destinados a la observación. Este crecimiento ha gatillado un aumento importante en la demanda de una nueva generaci\'on de m\'etodos que puedan procesar esta oleada de informaci\'on o big data astron\'omico.
\bigskip

Ejemplos de proyectos que actualmente producen una gran cantidad de datos a trav\'es de telescopios en diferentes partes del mundo son: el Panoramic Survey Telescope and Rapid Response System (Pan-STARRS)\cite{pan}, el Visible and Infrarred Survey Telescope (VISTA)\cite{vista}, el VLT Survey Telescope (VST)\cite{vst}, el Dark Energy Camera Legacy Survey (DECaLS)\cite{decals} y el Hyper Suprime-Cam Subaru Strategic Program (HSC SSP)\cite{ssp}. Estos surveys est\'an caracterizados por un amplio \textit{\'etendue} definido como el producto entre el \'area del espejo de un telescopio y su \'angulo s\'olido proyectado en el cielo.
\bigskip

En el futuro, telescopios como el Large Synoptic Survey Telescope (LSST)~\cite{lsst} (que entrará en funcionamiento a mediados del 2022) continuar\'an revolucionando la era del big data en astronom\'ia con \'etendues y c\'amaras CCD mucho m\'as grandes de lo que se utiliza hasta el d\'ia de hoy. En particular se espera que el LSST produzca un n\'umero de alertas de fen\'omenos transitorios (avisos de objetos que cambian en el tiempo o espacio) del orden de 10 millones cada noche. La capacidad de detectar nuevos objetos de inter\'es depender\'a de la calidad de los datos y de los algoritmos de tiempo real  destinados a generar las alertas mencionadas. 
\bigskip

Al d\'ia de hoy se han elaborado sondeos como el High Cadence Transient Survey (HiTS)~\cite{hits} cuya finalidad ha sido la b\'usqueda de fen\'omenos transitorios r\'apidos con escalas de tiempo que van desde las horas a d\'ias, utilizando secuencias de observaciones de la c\'amara DECam del telescopio Blanco (en Cerro Tololo) para la detecci\'on y posterior reporte de objetos candidatos a supernova. En este trabajo se propone un m\'etodo de detecci\'on de potenciales candidatos a supernova a trav\'es de la discriminaci\'on de p\'ixeles que involucren un incremento en la intensidad. Esta discriminaci\'on comienza con la determinaci\'on del flujo a trav\'es de la intensidad de los p\'ixeles que puedan corresponder a una estrella y a la variaci\'on de cada uno de ellos usando m\'etodos iterativos de filtrado en secuencias de im\'agenes de largo arbitrario. Los filtros desarrollados para HiTS corresponden a miembros de una familia de filtros conocidos como \textit{filtros de Kalman}.% En particular se hace uso de los filtros de Kalman b\'asico \cite{kalman} y de correntrop\'ia m\'axima \cite{chen}.
\bigskip


El trabajo desarrollado por Pablo Huentelemu~\cite{huentelemu} (M.Sc.), propone el uso de los filtros cl\'asico (o b\'asico) \cite{kalman} y de m\'axima correntrop\'ia \cite{chen} en el reconocimiento de supernovas j\'ovenes (o en su fase de crecimiento de luminosidad) de tipo II, por lo que se plantea la posibilidad de dise\~nar alg\'un otro criterio de filtrado y de estudiar la variaci\'on de los resultados.
\bigskip

%En el presente trabajo se pretende desarrollar un nuevo m\'odulo de filtro de Kalman que sea robusto a sistemas no-lineales para la detecci\'on de potenciales candidatos a supernova, para posteriormente ser embebido en un programa que pueda funcionar en l\'inea generando alarmas de detecci\'on. Por otro lado se busca implementar un software que contenga los m\'etodos de filtrados originales (versiones cl\'asica y de m\'axima correntrop\'ia del filtro de Kalman) a trav\'es de un refactoring del programa  base. Paralelamente, se desea encontrar rangos apropiados para los umbrales requeridos para el filtrado.


\section{Objetivos}
\subsection{Objetivo general}
El objetivo de esta tesis comprende la reestructuraci\'on y extensi\'on de un programa existente destinado al an\'alisis fotom\'etrico de datos astron\'omicos con \'el cual se busca implementar un modelo de sistema de alertas que d\'e aviso del estado temprano de un fen\'omeno astron\'omico transitorio. 
\bigskip

\subsection{Objetivos espec\'ificos}

\begin{itemize}


\item Mejorar proceso existente sobre la manipulaci\'on de los archivos requisito del programa original, debido a que actualmente existen problemas de \textit{hard-coding} respecto de la ubicaci\'on de estos, as\'i como ciertos reparos durante el proceso de selecci\'on de los mismos. Adem\'as se necesita reformular la configuraci\'on de par\'ametros funcionales de entrada (propios de la ejecuci\'on del programa, como los umbrales) para flexibilizar su entrada. 
\bigskip

\item Generar una familia de m\'etodos en t\'erminos de programaci\'on orientada a objetos empleando el patr\'on de dise\~no \textit{Strategy} que implemente la familia de m\'etodos conocida como Filtros de Kalman.
\bigskip

\item Agregar una nueva variante de filtro que permita el uso de un modelo no lineal.
\bigskip

\item Implementar gr\'aficas de resultados en las que sea posible observar la curva generada por el flujo y la velocidad de flujo estimados por los filtros, indicando la entrop\'ia aproximada de la curva en el espacio de estados y visualizar la evoluci\'on de las mediciones y estimaciones tanto en secuencias de im\'agenes estampillas (p\'ixeles) como en gr\'aficas.    
\bigskip

\item Estudiar los resultados obtenidos con esta nueva versi\'on del programa con cada uno de sus filtros sobre un conjunto de datos brindado por HiTS (del a\~no 2015), a la par que el desempe\~no (medido en t\'erminos de uso de memoria principal y tiempo de ejecuci\'on) de \'este. Por \'ultimo se busca establecer un contraste con la versi\'on original.     

\end{itemize}
\section{Organizaci\'on de la tesis}

Este documento contiene los siguientes cap\'itulos: en el Cap\'itulo \ref{ch:background} se describe en qu\'e consiste una supernova, el proyecto HiTS  as\'i como el background matem\'atico de los m\'etodos de filtrado usados en el trabajo original (filtros de Kalman b\'asico y de m\'axima correntrop\'ia) y el filtro a implementar (filtro de Kalman unscented).%la relevancia de la linealidad de un sistema (y su impacto en la estad\'istica del mismo en la aplicaci\'on de un m\'etodo de filtrado como el filtro de Kalman)%
\bigskip

%El siguiente cap\'itulo, \ref{ch:linear} est\'a destinado para explicitar la relevancia de la linealidad de un modelo y que ocurre en casos de que este no sea, al aplicar un filtro de Kalman.
%\bigskip

En el Cap\'itulo \ref{ch:prev_work}, se estudian tanto el desempe\~no como los resultados obtenidos con el c\'odigo original del programa sobre el cual se trabajar\'a, estableci\'endose una breve discusi\'on sobre estos resultados. 
\bigskip

Posteriormente, en el Cap\'itulo \ref{ch:refactoring} se describe el refactoring del programa  del c\'odigo original: se enlistan nuevos m\'etodos para el manejo de archivos y cambios realizados en la implementaci\'on de los m\'etodos iniciales. 
\bigskip

En el Cap\'itulo \ref{ch:news} se exponen los pasos para desarrollar el nuevo filtro, adem\'as de la nueva funcionalidad relacionada con la carga de resultados previamente guardados.
\bigskip

Posteriormente, en el Cap\'itulo \ref{ch:resultados}, se exponen los resultados obtenidos para el desempe\~no de la nueva versi\'on del programa (para los dos filtros originales) incluyendo la cantidad de supernovas redescubiertas y el per\'iodo en que se detectaron. El cap\'itulo concluye con el an\'alisis de los resultados y una comparaci\'on con aquellos obtenidos en el Cap\'itulo \ref{ch:prev_work}.
\bigskip

Finalmente en el Cap\'itulo \ref{ch:conclusion} se presentan las conclusiones de este trabajo y se propone el trabajo a futuro a realizar.

%\fillup{1}

%\todo{todo 1} asdfasfsdfsadafs \missingref{me faltaría una ref!}.
%\todo[inline]{todo 2} 
\chapter{Antecedentes}
\label{ch:background}
En el presente Cap\'itulo se exponen conceptos cruciales para la comprensi\'on y desarrollo de este trabajo. Se describe en particular los objetos de inter\'es que estimularon el desarrollo del survey HiTS: las supernovas de tipo II y el evento de shock-breakout asociado a \'estas y una breve descripci\'on de los datos obtenidos por este proyecto durante el a\~no 2015. De la misma forma, se describe la base matem\'atica en la que se basan las diferentes versiones del filtro de Kalman implementadas en el software original, es decir, los filtros de Kalman \textit{b\'asico} y de \textit{m\'axima correntrop\'ia} as\'i como una nueva versi\'on del m\'etodo conocida como \textit{unscented} que permitir\'ia la introducci\'on de modelos no lineales en el proceso de filtrado.  

\section{Supernova tipo II}\label{sec:sn}
Una supernova de tipo II corresponde a un evento estelar con el que finaliza la vida de una estrella masiva (aquellas que en su proceso de formaci\'on poseen una masa superior a 10 masas solares \footnote{$M_{\odot} = (1.98847 \pm 0.00007) \times 10^{30}$ Kg}). Al no contar con combustible necesario para llevar a cabo reacciones nucleares que puedan contrarrestar su propia gravedad (su n\'ucleo ya no puede formar elementos m\'as pesados que el hierro o el n\'iquel, los cuales caen al n\'ucleo de la estrella por su propio peso. Ver Figura \ref{fig:f0}.), y si la presi\'on degenerada\footnote{presi\'on que viene del principio de exclusi\'on de Pauli} de los electrones del plasma de la estrella no es suficiente para soportar este peso, la estrella se contrae abruptamente incrementando la temperatura de su centro a $10^{10}$ K.
\bigskip

\begin{figure}[h!]
\centering
\includegraphics[scale=.25]{images/sncore}
\caption{Esquema no a escala de la estructura de una estrella masiva previo a su explosi\'on como supernova. Los elementos m\'as pesados se alojan en el centro, mientras que  los m\'as livianos, como el hidr\'ogeno o el helio lo hacen en la capa m\'as externa. \textit{Imagen publicada por R. J. Hall en WikiMedia Commons, 15 de agosto de 2007.}}
\label{fig:f0}
\end{figure}


Debido al aumento de temperatura, los electrones del n\'ucleo adquieren energ\'ia cin\'etica suficiente para escapar, desapareciendo as\'i la presi\'on que ejerc\'ian hacia el exterior. Finalmente el n\'ucleo colapsa liberando energ\'ia gravitacional y con ella las capas m\'as externas de la estrella son expulsadas en una gran explosi\'on. Este fen\'omeno se denomina \textit{supernova} e implica un aumento repentino del brillo de una estrella, incrementando su brillo en un factor de $10^8$ veces, pudiendo incluso ser m\'as brillante que la galaxia que la alberga.
\bigskip

En general la variaci\'on de la luminosidad de una supernova corresponde a una curva que crece r\'apidamente los primeros d\'ias (u horas), alcanzando un m\'aximo, para luego decaer. Cabe destacar que existen otros tipos de supernova, c\'omo las de las de tipo Ia y corresponden a otro fen\'omeno en donde participa una clase de estrella denominada enana blanca junto a otra estrella de cualquier otro tipo. La primera, al poseer una gravedad tan alta en su superficie es capaz de tomar material de su compa\~nera con lo que al superar las 1.44 $M_{\odot}$ se desencadenar\'ia la explosi\'on de supernova. 
\bigskip

Las supernovas de tipo Ia presentan un decaimiento casi continuo una vez alcanzado el m\'aximo, mientras que las de tipo II presentan dos ca\'idas: una inmediatamente despu\'es de su m\'aximo y otra una vez finalizado un per\'iodo de decaimiento suavizado (figura ~\ref{fig:f1}). Otra forma de diferenciarlas es la presencia de trazas de hidr\'ogeno en los espectros de \'estas: las supernovas de tipo Ia pr\'acticamente no presentan hidr\'ogeno (l\'ineas de absorci\'on distintivas) a diferencia de las de tipo II.\bigskip

\begin{figure}[h!]
\centering
\includegraphics[scale=.8]{images/clear}
\caption{Curvas t\'ipicas de supernovas Ia (enana blanca) y II (estrella masiva). \textit{\textcopyright 2004 Pearson Education Inc., publishing as Addison-Wesley The Bizarre Stellar Graveyard.}}
\label{fig:f1}
\end{figure}


\section{High Cadence Transient Survey: HiTS}

El High Cadence Transient Survey\cite{hits} (desde ahora, HiTS) es un survey cuyo objetivo principal es detectar y seguir fen\'omenos transitorios estelares en escalas de tiempo que van desde horas a d\'ias, con especial atenci\'on a fases tempranas de explosiones de supernovas (primeras horas): el objetivo original de HiTS corresponde a la detecci\'on de un fen\'omeno llamado \textit{shock breakout} (SBO), un fen\'omeno que ocurre inmediatamente despu\'es del colapso del n\'ucleo de una estrella roja supergigante (una de las posibles etapas finales de una estrella masiva antes de \textit{explotar} en supernova II). Ver figura \ref{fig:f2}\footnote{$L_{\odot}= 3.828 \times 10^{26}$ W}.

%\ref{fig:f2}\footnote{\url{https://www.nasa.gov/feature/ames/Kepler/caught-for-the-first-time-the-early-flash-of-an-exploding-star}}


\begin{figure}[h!]
\centering
\includegraphics[scale=.25]{images/breakout}
\caption{Diagrama que ilustra la evoluci\'on del brillo de una supernova en t\'erminos de luminosidad solar ($L_{\odot}$) durante d\'ias. Se resalta el fen\'omeno de \textit{shock-breakout} apenas comienza el incremento de la luminosidad de la supernova. Esta imagen fue publicada en la p\'agina de la NASA destacando la primera vez que un evento como este es \textit{capturado} en la banda visible (por el telescopio espacial Kepler). \textit{NASA Ames/W. Stenzel. 2016}}
\label{fig:f2}
\end{figure}

HiTS utiliza la Dark Energy Camera (DECam, figura \ref{fig:f3}) para la obtenci\'on de sus im\'agenes. Esta c\'amara se encuentra montada en el Telescopio Blanco del Observatorio de Cerro Tololo (CTIO) en la regi\'on de Coquimbo, Chile. Esta c\'amara posee 62 detectores CCD de $2048 \times 4096$ p\'ixeles para la obtenci\'on de im\'agenes cient\'ificas y otros 12 para la gu\'ia, alineamiento y enfoque (ver Figuras \ref{fig:f3} y \ref{fig:f4} d\'onde se muestra disposici\'on de las c\'amaras CCD en el telescopio). 
\bigskip

\begin{figure}[h!]
\centering
\includegraphics[scale=.5]{images/CCDs.jpg}
\caption{A la izquierda, estructura de la c\'amara DECam poblada con 62 chips CCDs. A la derecha, imagen \textit{flat field} desde DECam. \textit{Im\'agenes tomadas desde la p\'agina del Dark Energy Survey (\url{www.darkenergysurvey.org/the-des-project/instrument/the-camera}).}}
\label{fig:f3}
\end{figure}

Durante el proyecto HiTS se realizaron tres campa\~nas de observaci\'on en los a\~nos 2013, 2014 y 2015 durante el primer semestre de cada a\~no. Se escogieron 40 campos y 4 \'epocas por noche (tambi\'en por campo) para observaciones para el a\~no 2013 y en el 2014. Para la campa\~na del a\~no 2015 se escogieron 50 campos. En estas campa\~nas se obtuvieron m\'as de 120 candidatos a supernova. Sin embargo, no se logr\'o encontrar en estas rastros de SBO (figura \ref{fig:f2}) evento que comprendi\'o uno de los principales objetivos de HiTS. 

\begin{figure}[h!]
\centering
\includegraphics[scale=.75]{images/decam}
\caption{Orientaciones sobre el cielo y la huella espacial del arreglo de detectores en el plano focal. Se destacan los CCD cuya etiqueta comienzan con S o N, ya que estos corresponden a los detectores encargados de obtener las im\'agenes cient\'ificas. El etiquetado de estas componentes puede ser enga\~noso debido a que las iniciales de norte (North) y sur (South) est\'an invertidas en relaci\'on a la orientaci\'on del cielo. \textit{Imagen publicada en el sitio del CTIO (\url{http://www.ctio.noao.edu/noao/node/2250})}.}
\label{fig:f4}
\end{figure}

\begin{figure}
\centering
\includegraphics[scale=.5]{images/fields}
\caption{Distribuci\'on espacial de los campos observados durante los primeros semestres de los a\~nos 2013 (gris), 2014 (azul) y 2015 (naranjo). En tono rojo, los mismos campos del a\~no 2015 y 2014 (superposici\'on). \textit{F. F\"orster et al., 2015. HiTS real-time supernova detections.}}
\end{figure}

\subsection{Datos obtenidos durante el a\~no 2015}\label{ssec:data}
Como se mencion\'o anteriormente, la campa\~na del a\~no 2015 tuvo lugar el primer semestre de ese a\~no. El per\'iodo en que se llev\'o a cabo fue durante los meses de febrero y marzo; espec\'ificamente entre los d\'ias 17 de febrero a 14 de marzo.
\bigskip

El per\'iodo comprendido por los d\'ias 17 a 22 de febrero fue el de mayor latencia, obteni\'endose cinco \'epocas (observaciones) por noche por cada campo. Posteriormente la toma de observaciones cesa y se reanuda a partir del 24 de febrero con una latencia de a lo m\'as tres \'epocas por noche y campo con interrupciones de hasta diez d\'ias finalizando el 14 de marzo.

\section{El filtro de Kalman}
La evoluci\'on determin\'istica de un sistema f\'isico en el tiempo es conocida si el estado del sistema es medido con absoluta precisi\'on en cada instante de tiempo (i.e., en un entorno donde es posible despreciar fen\'omenos cu\'anticos). Sin embargo toda medici\'on est\'a sujeta a incertezas finitas. Para sistemas los cuales son observados entre intervalos prolongados de tiempo, se prevee que las diferencias entre los estados estimados y los medidos se incrementen con el tiempo. Para la obtenci\'on de predicciones lo m\'as confiables posible se requiere que el sistema sea regularmente monitoreado y sus estados estimados puedan ser considerados confiables en un lapso de tiempo apropiado. 
\bigskip

Los filtros de Kalman son m\'etodos que proveen un compromiso (o trade-off) entre los valores esperados del estado actual de un sistema y las mediciones que proporcionan informaci\'on de su estado real. La aplicaci\'on de un filtro de Kalman est\'a pensada como un proceso de dos fases:
\begin{enumerate}
\item \textbf{Fase predictiva:} Una estimaci\'on del estado actual del sistema que se basa en la estimaci\'on del estado previo o \'ultimo estado. Esta predicci\'on se denomina usualmente como estado estimado \textit{a priori}. 
\item \textbf{Fase correctiva:} La estimaci\'on del estado \textit{a priori} es corregida con una medida actual para refinar la aproximaci\'on. Esta mejora se denomina \textit{aproximaci\'on a posteriori.}
\end{enumerate}

T\'ipicamente estas fases de predicci\'on y correcci\'on se van alternando mientras se estudia el comportamiento f\'isico de alg\'un sistema.
\bigskip

\subsection{Filtro de Kalman B\'asico}
El filtro de Kalman B\'asico \cite{kalman} asume un comportamiento de sistema lineal y que las mediciones y las predicciones siguen una distribuci\'on Gaussiana. 
\bigskip

A continuaci\'on se describen las componentes del desarrollo matem\'atico del filtro:

\begin{itemize}
\item \textbf{$F_k$:} Matriz de transici\'on de estado, de dimensiones $N\times N$ (N es el n\'umero de variables de estado).
\item \textbf{$H_k$:} Matriz de transformaci\'on de estado a medici\'on, $K\times N$ (K corresponde a las mediciones realizadas en un instante k de una variable de estado).
\item \textbf{$Q_k$:} Matriz de covarianza del ruido del proceso ($N\times N$).
\item \textbf{$R_k$:} Matriz de covarianza del ruido de las mediciones ($K\times K$).
\item \textbf{$B_k$:} Matriz de control de entrada (contiene alteraciones que se querr\'ian agregar al sistema de manera deliberada, por ejemplo, como la condici\'on de parada de un veh\'iculo en movimiento). Esta matriz es de dimensiones $N\times L$, donde $L$ es la dimensi\'on del vector de control de entrada $u_k$.
\end{itemize}
\bigskip
Se har\'a uso de la notaci\'on en sub\'indices $m|n$, en las estimaciones de estado y covarianzas, para explicitar el instante de tiempo al cual pertenecen:  $m$; y al instante de tiempo de donde se extrae la informaci\'on: $n$.
\bigskip

En el instante $k-1$, se obtienen dos cantidades 
\begin{itemize}
\item $\hat{x}_{k|k-1}$ : Estado estimado \textit{a priori} (vector de dimensi\'on $N$). 
\item $P_{k|k-1}$ : Matriz de covarianza \textit{a priori} (matriz de dimensi\'on $N\times N$ ).
\end{itemize}
\bigskip

Luego, en la fase de correcci\'on se calculan:

\begin{itemize}
\item $\hat{z}_k$ : Predicci\'on de la medida.
\item $\tilde{z}_k$ : Residuo de la diferencia entre la medici\'on real, $z_k$, y su predicci\'on, $\hat{z}_k$.
\item $\hat{x}_{k|k}$ : Estado estimado \textit{a posteriori}.
\item $P_{k|k}$ : Matriz de covarianza \textit{a posteriori}.
\end{itemize}
\bigskip

Con estas variables, podemos describir las Ecuaciones que explican la evoluci\'on del proceso del algoritmo de Kalman b\'asico (un esquema de los c\'alculos involucrados en el proceso de estimaci\'on de estados se puede ver en la figura \ref{fig:kfb}):
\begin{enumerate}
\item \textbf{Fase predictiva:}\\
Las ecuaciones de estimaci\'on de estado y matriz de covarianza \textit{a priori} son:
\begin{equation}
\hat{x}_{k|k-1} = F_k \hat{x}_{k-1|k-1} + B_k u_k,
\label{eq:eq1}
\end{equation}
\begin{equation}
P_{k|k-1} = F_{k}P_{k-1|k-1}F_k^{T} + Q_k. 
\label{eq:eq2}
\end{equation}
\item \textbf{Fase correctiva:}\\
Las ecuaciones de stimaci\'on de estado y matriz de covarianza \textit{a posteriori} son:
\begin{equation}
\hat{z}_k = H_{k} \hat{x}_{k|k-1},
\label{eq:eq3}
\end{equation}
\begin{equation}
\tilde{z}_k=z_k - \hat{z}_k.
\label{eq:eq4}
\end{equation}
La Ecuaci\'on \ref{eq:eq4} describe la obtenci\'on de un residuo de la diferencia entre la predicci\'on y la medida $z_k$. Posteriormente se calcula la matriz de covarianza entre residuos ($S_k$), con la que se calcula la ganancia de Kalman: $K_k$ . Formalmente,
\begin{equation}
S_k = H_k P_{k|k-1} H_{k}^T + R_k,
\label{eq:eq5}
\end{equation}
\begin{equation}
K_k = P_{k|k-1} H_k^T S_k^{-1}.
\label{eq:eq6}
\end{equation}
Con la ganancia de Kalman calculada, se actualiza el valor de la estimaci\'on de estado (\ref{eq:eq7}) y la matriz de covarianza a posteriori (Ecuaci\'on \ref{eq:eq8}), del siguiente modo, 

\begin{equation}
\hat{x}_{k|k} = \hat{x}_{k|k-1} + K_k \tilde{z}_k,
\label{eq:eq7}
\end{equation}

\begin{equation}
P_{k|k} = (I_N - K_kH_k)P_{k|k-1}.
\label{eq:eq8}
\end{equation}
\bigskip

\end{enumerate}
La figura \ref{fig:kfb} resume el proceso de predicci\'on (obtenci\'on de las cantidades a priori, $\hat{x}_{k|k-1}$ y $P_{k|k-1}$) y correcci\'on (generaci\'on de las estimaciones, $\hat{x}_{k+1|k}$ y $P_{k+1|k}$, a partir de las cantidades a posteriori  $\hat{x}_{k|k}$ y $P_{k|k}$) del filtro de Kalman B\'asico. 
\begin{figure}[h!]
\centering
\includegraphics[scale=0.5]{images/kfb}
\caption{Representaci\'on del proceso de predicci\'on (obtenci\'on de cantidades a priori) de las cantidades $\hat{x}_{k|k-1}$ y $P_{k|k-1}$;  y de correcci\'on (estimaci\'on a posteriori) para obtener las cantidades $\hat{x}_{k+1|k}$ y $P_{k+1|k}$. \textit{E. Matsinos, 2016. The Kalman Filter: a didactical overview.}}
\label{fig:kfb}
\end{figure}

\subsection{Filtro de Kalman de M\'axima Correntrop\'ia}

El filtro de Kalman basado en m\'axima correntrop\'ia \cite{badong}, difiere del filtro de Kalman tradicional (b\'asico) en que no asume gaussianidad en las observaciones, considerando casos en que una se\~nal puede ser perturbada por pulsos de ruido que sigan una distribuci\'on de cola pesada. En esta oportunidad se utiliza el \textit{criterio de m\'axima correntrop\'ia} para el proceso de correcci\'on. 
\bigskip

La correntrop\'ia es una medida de similitud entre dos variables aleatorias. Supongamos, $X,Y \in \mathbb{R}$ con una distribuci\'on conjunta $F_{XY} (x,y)$. Definimos la correntrop\'ia matem\'aticamente como:

\begin{equation}
V(X,Y) = E[\kappa(X,Y)] = \int \kappa(x,y) dF_{XY} (x,y),
\label{eq:eqcorr}
\end{equation}
\noindent
donde $E$ representa al operador de esperanza y $\kappa(,)$ corresponde a un kernel Mercer invariante a desplazamientos (teorema de Mercer, \cite{mercer}). Para este filtro se emplea una funci\'on de kernel Gaussiana, dado por

\begin{equation}
\kappa(x, y) = G_{\sigma} (e) = exp \left(-\dfrac{e^2}{2\sigma^2} \right).
\label{eq:eqkappa}
\end{equation}

La \textit{funci\'on de costo} de m\'axima correntrop\'ia se define como:
\begin{equation}
J_{MCC} = \dfrac{1}{N} \sum_{i=1}^N G_{\sigma} (e (i)).
\label{eq:costo}
\end{equation}
La Ecuaci\'on \ref{eq:costo} representa la funci\'on a maximizar, la que calcula la estimaci\'on corregida para el estado en el instante $k$,

\begin{equation}
\hat{x}_k = arg max_{x} (J_{MCC})= argmax_x (\sum_{i=1} G_{\sigma} (e_i(k))).
\label{eq:max}
\end{equation}
\bigskip

Para obtener el m\'aximo de correntrop\'ia se procede a calcular el error residual $\tilde{e}_i$ estimado como,

\begin{equation}
\tilde{e}_i = d_{i,k} - w_{i,k} \hat{x}_{t-1, k|k}.
\label{eq:eq9}
\end{equation}

Con estos residuos definimos las matrices diagonales, descritas en las Ecuaciones siguientes: 
\bigskip

\begin{equation}
\tilde{C_{x, k}}= diag(G_{\sigma}(\tilde{e}_{1, k}),..., G_{\sigma}(\tilde{e}_{n, k})),
\label{eq:eq10}
\end{equation}

\begin{equation}
\tilde{C_{y, k}}= diag(G_{\sigma}(\tilde{e}_{n+1, k}),..., G_{\sigma}(\tilde{e}_{n+m, k})).
\label{eq:eq11}
\end{equation}

La matriz \ref{eq:eq10} corresponde a evaluaciones del kernel en errores estimados de predicci\'on y la matriz \ref{eq:eq11} en errores propios de las observaciones (ruido). Luego se realiza una transformaci\'on de la covarianza del ruido de las mediciones usando una descomposici\'on de Choleski ($B_{r, k}$) en el instante $k$ \cite{chen}, seg\'un la expresi\'on:
\bigskip
 
\begin{equation}
\tilde{R}_k = B_{r, k} \tilde{C}_{y, k}^{-1}B_{r, k|k-1}^T.
\label{eq:eq12}
\end{equation}

Luego se calcula la transformaci\'on de la predicci\'on de la matriz de covarianza de las estimaciones de estado, $P_{k|k-1}$,
\begin{equation}
\tilde{P}_{k|k-1} = B_{p, k|k-1} \tilde{C}_{x, k}^{-1}B_{p, k|k-1}^T.
\label{eq:eq13}
\end{equation}

Posteriormente se calcula la ganancia de Kalman para este nuevo sistema,
\begin{equation}
\tilde{K} = \tilde{P}_{k|k-1} H_k^T (H_k \tilde{P}_{k|k-1} H_k^T + \tilde{R}_k)^{-1}. 
\label{eq:eq14}
\end{equation}

Finalmente la actualizaci\'on de la estimaci\'on de estado para el instante $k$ queda como: 
\begin{equation}
\hat{x}_{t, k|k} = \hat{x}_{k|k-1} + \tilde{K}_{k} (y_k - H_k \hat{x}_{k|k-1}).
\label{eq:eq15}
\end{equation}
\bigskip

Las Ecuaciones \ref{eq:eq9} a \ref{eq:eq15} se repiten secuencialmente hasta satisfacer la condici\'on:

\begin{equation}
\dfrac{\parallel  \hat{x}_{t, k|k} - \hat{x}_{t-1, k|k} \parallel }{\parallel \hat{x}_{t-1, k|k} \parallel} \leq \epsilon.
\label{eq:eq16}
\end{equation}

El valor de $\epsilon$ es definido por el usuario y corresponde a un criterio de detenci\'on (el algoritmo tambi\'en puede detenerse definiendo un m\'aximo de n\'umero de pasos). Finalmente se calcula la matriz de covarianza para el instante actual, $k$.
%\item \textbf{C\'alculo de la m\'axima correntrop\'ia basado en funci\'on costo}
\begin{equation}
P_{k|k} = \left(I - \tilde{K}_kH_k \right)P_{k|k-1} \left(I - \tilde{K}_k H_k \right)^T + \tilde{K}_kR_k\tilde{K}^T_k.
\label{eq:eq17}
\end{equation}

\subsection{Filtro de Kalman Unscented (UKF)}
\label{ssec:ukf}
Este filtro corresponde a aquel que se pretende agregar a la familia de filtros ya desarrollada en el programa base.
\bigskip

\begin{enumerate}
\item \textbf{Fase predictiva:}\\
En esta versi\'on del filtro \cite{ukf} ya no se habla de matrices de transici\'on de estado, $F_k$, ni de matrices de transformaci\'on de estado-a-medici\'on, $H_k$, sino m\'as bien de funciones diferenciables $f$ y $h$ respectivamente para describir la transici\'on de estados  y la transformaci\'on de estos a estimaciones a priori. Sin embargo, previo a estas transiciones se deben seleccionar 2N+1 puntos representativos alrededor de $\hat{x}_{k-1|k-1}$ y evaluar estos en la funci\'on no lineal $f$, para obtener las estimaciones de $\hat{x}_{k|k-1}$ y $P_{k|k-1}$. Estos puntos se conocen como \textit{puntos sigma} (en la Figura \ref{fig:fukf} se visualiza el proceso de predicci\'on al momento de obtener el primer conjunto de \textit{puntos sigma}, propagarlos usando la funci\'on $f$ y obtener $\hat{x}_{k|k-1}$ y $P_{k|k-1}$, es decir, la matriz de estados y de covarianza \textit{a priori}). 

\begin{figure}[h!]
\includegraphics[scale=.5]{images/ukf}
\caption{Representaci\'on del funcionamiento del filtro UKF. En esta oportunidad se hace uso de la funci\'on $f$ y $h$ para obtener las transformaciones $x_{k-1}\rightarrow x_k$ y $x_{k}\rightarrow z_k$. Esto se logra con la evaluaci\'on de los 2N+1 \textit{puntos sigma} generados durante la etapa de predicci\'on (y posteriormente en la etapa de correcci\'on). \textit{E. Matsinos, 2016. The Kalman Filter: a didactical overview.}}
\label{fig:fukf} 
\end{figure}

La generaci\'on de los 2N+1 puntos, se realiza a partir de la \'ultima estimaci\'on $\hat{x}_{k-1|k-1}$  de la siguiente forma:
\begin{equation}
\label{eq:eq18}
\begin{gathered}
\bar{x}_{k-1| k-1}^0 = \hat{x}_{k-1|k-1}\\
\bar{x}_{k-1| k-1}^0 = \hat{x}_{k-1|k-1}+ \chi_i, \quad  \forall i \in [1, N]\\
\bar{x}_{k-1| k-1}^0 = \hat{x}_{k-1|k-1}- \chi_{i-N}, \quad  \forall i \in [N+1, 2N]
\end{gathered},
\end{equation}
donde la cantidad $\chi_i$ corresponde a la i-\'esima columna de la \textit{ra\'iz cuadrada} de la matriz:

 \begin{equation}
 (N+\lambda) P_{k-1 | k-1}.
 \label{eq:eq19}
 \end{equation}
La matriz (\ref{eq:eq19}) puede obtenerse a partir de la descomposic\'on de Choleski. Por otro lado los puntos sigma se generan junto a dos conjuntos de pesos: $\lbrace w_x^{i} \rbrace$ y $\lbrace w_p^{i} \rbrace$. El primer conjunto se emplea en la estimaci\'on del estado y la predicci\'on de la medida, mientras que el segundo conjunto es usado para obtener las matrices de covarianza. Estos pesos son definidos como:
\begin{equation}
\label{eq:eq20}
\begin{gathered}
w^0_x = \dfrac{\lambda}{N+\lambda}\\
w^0_p = w^0_x + 1 - \alpha^2 + \beta\\
w^i_x = w^i_p = \dfrac{1}{2(N+\lambda)}\\
\sum_i^{2N} w^i_x = 1
\end{gathered}.
\end{equation}
De \ref{eq:eq20} se desprende que los pesos $w_x^i$ son normalizados. Por otro lado, el par\'ametro $\lambda$ se puede escribir en t\'erminos de los valores de $\alpha \in \left( 0,1\right]$ y $\kappa$ seg\'un la expresi\'on siguiente:%\ref{eq:eq21}

\begin{equation}
\label{eq:eq21}
\lambda = \alpha^2  (N + \kappa)- N.
\end{equation}

Los par\'ametros $\alpha$, $\beta$ y $\kappa$ deben ser ajustados acorde al problema que se est\'a estudiando.
\bigskip

Con esto, es posible escribir las Ecuaciones de la fase predictiva.
\begin{itemize}
\item Estimaci\'on a priori de los estados. La ecuaci\'on correspondiente es:\\
\begin{equation}
\label{eq:eq22}
\hat{x}_{k|k-1} = \sum_{i=0}^{2N} w_{x}^i f(\bar{x}^i_{k-1|k-1}).
\end{equation}

\item Estimaci\'on a priori de la matriz de covarianza. La ecuaci\'on correspondiente es:\\

\begin{equation}
\label{eq:eq23}
P_{k|k-1} = \sum_{i=0}^{2N} w_p^i \left( f(\bar{x}^i_{k-1|k-1})  - \hat{x}_{k|k-1}\right)\left( f(\bar{x}^i_{k-1|k-1}) - \hat{x}_{k|k-1}  \right)^T + Q_k.
\end{equation}
\end{itemize}


\item \textbf{Fase correctiva:}\\
Durante la fase de correcci\'on, nuevamente se seleccionan 2N+1 puntos representativos, alrededor de $\hat{x}_{k|k-1}$. Estos posteriormente son evaluados en la funci\'on no-linear $h$.

\begin{equation}
\label{eq:eq24}
\begin{gathered}
\bar{y}_{k-1| k-1}^0 = \hat{x}_{k|k-1}\\
\bar{y}_{k-1| k-1}^i = \hat{x}_{k|k-1}+ \psi_i, \quad  \forall i \in [1, N]\\
\bar{y}_{k-1| k-1}^i = \hat{x}_{k|k-1}- \psi_{i-N}, \quad  \forall i \in [N+1, 2N].\\
\end{gathered}
\end{equation}
La cantidad $\psi_i$ representa la i-\'esima columna de la matriz de \textit{ra\'iz cuadrada} $(N+\lambda)P_{k|k-1}$.

Las Ecuaciones del proceso de correcci\'on, por tanto, quedan como sigue:
\begin{itemize}
\item Predicci\'on de las medidas:\\
\begin{equation}
\label{eq:eq25}
\hat{z}_{k} = \sum_{i=0}^{2N} w_x^i h(y_{k|k-1}^{-i}).
\end{equation}
\item Los residuos de las mediciones pueden obtenerse como:\\
\begin{equation}
\tilde{z}_k = z_k - \hat{z}_k.
\label{eq:eq26}
\end{equation}
\item La matriz de innovaci\'on:\\
\begin{equation}
S_k = \sum_{i=0}^{2N} w_p^i (h(y_{k|k-1}^{-i}) - \hat{z}_k)(h(y_{k|k-1}^{-i})^T + R_k.
\label{eq:eq27}
\end{equation}
\item La matriz de covarianza cruzada de estado a medida se describe como:\\
\begin{equation}
C_k = \sum_{i=0}^{2N} w_p^i ( f(\bar{x}^i_{k-1 | k-1})- \hat{x}_{k|k-1} )( h(y_{k|k-1}^{-i}) - \hat{z}_k )^T.
\label{eq:eq28}
\end{equation}
\item La ganancia \'optima finalmente queda:\\
\begin{equation}
K_k = C_kS_k^{-1}.
\label{eq:eq29}
\end{equation}
\item La estimaci\'on \textit{a posteriori} de estado:\\
\begin{equation}
\label{eq:eq30}
 \hat{x}_{k|k} =  \hat{x}_{k|k-1} + K_k \tilde{z}.
\end{equation}
\item Por otro lado, la ecuaci\'on para la matriz de covarianza:\\
\begin{equation}
\label{eq:eq31}
P_{k|k} = P_{k|k-1} - K_kS_kK_k^T.
\end{equation}
\end{itemize}
Los pesos $w_x^i$ y $w_p^i$ son los mismos calculados en la expresi\'on \ref{eq:eq20}, de la fase de predicci\'on.
\end{enumerate}

\section{Laboratorio Nacional de Computaci\'on de Alto Desempe\~no (NLHPC)}
El National Laboratory for High Performance Computing (NLHPC) es un proyecto asociativo, financiado por el PIA de CONICYT el cual dispone de un potente sistema computacional que est\'a disponible a la comunidad cient\'ifica y acad\'emica nacional (instituciones de investigaci\'on, industria y universidades), estimulando su uso en el desarrollo de \'areas de investigaci\'on que requieran de herramientas computacionales robustas que deban ser usadas de manera intensiva. 
\bigskip

Para la realizaci\'on de esta tesis se hizo uso de uno de los nodos del cl\'uster de Leftraru, el supercomputador del NLHPC disponible para la comunidad investigadora desde el 2014 en las instalaciones del Centro de Modelamiento Matem\'atico (CMM) de la Universidad de Chile.  

\begin{itemize}
\item 132 nodos de c\'omputo HP (128 nodos HP SL230 y 4 nodos HP SL250), cada uno con dos procesadores de 10 cores Intel Xeon Ivy Bridge E5-2660 V2.
\item 2640 n\'ucleos
\item 6.25 TB de RAM
\item 274TB de almacenamiento Lustre (DDN EXAScaler)
\item 12 co-procesadores Intel Xeon Phi5110p de 2 TFlops
\item Capacidad de c\'omputo de 70 TFlops 
\end{itemize} 

Para la ejecuci\'on de las pruebas se emplearon cuatro cores, 2400 MB por cada CPU (m\'axima RAM permitida) y un \textit{job-array} de largo 93.


\include{25_linearity}
\chapter{Evaluaci\'on del programa original}
\label{ch:prev_work}

A continuaci\'on se ofrece una breve descripci\'on de la rutina original y se exponen los resultados de diferentes pruebas realizadas con ella, con la finalidad de medir su desempe\~no computacional en t\'erminos de tiempo de ejecuci\'on y uso de memoria, a la par de estudiar los resultados en el proceso de detecci\'on de las 93 supernovas halladas por HiTS durante la campa\~na del a\~no 2015.
\bigskip


\section{El programa}

La pipeline original est\'a estructurada por un proceso que puede dividirse en dos bloques: el primero est\'a destinado a buscar una supernova confirmada por HiTS (cuya coordenada es conocida) y encontrar nuevos candidatos; mientras que el segundo bloque, con la lista de coordenadas de nuevos candidatos, est\'a destinado a obtener la informaci\'on\footnote{Con informaci\'on se refiere a extraer muestras o estampillas desde las im\'agenes (de flujo, cient\'ifica, etc) como tambi\'en de las estructuras matriciales (matrices de estado, covarianza asociada, etc.) centradas en la coordernada del respectivo candidato.} de estos. El proceso general se inicializa con una instancia de \textsc{RunData}, la cual posee objetos de clases como \textsc{KalmanFilter} y \textsc{SNDetectos} (con los cuales se configurar\'a la rutina), y con la cual se establecer\'an los par\'ametros a emplear a trav\'es de \textit{hardcoding}. Posterior a la configuraci\'on de par\'ametros se tiene una instancia de \textsc{FITSHandler}, para preparar las im\'agenes y el resto de archivos que ser\'an usados para el proceso iterativo que se describe a continuaci\'on:
\bigskip


\begin{enumerate}
\item \textbf{C\'alculo de flujo:} A partir de las im\'agenes cient\'ificas y de calibraci\'on se obtiene el flujo (en ADU\footnote{Analog-to-digital unit}) por p\'ixel y es registrado en una matriz (\texttt{numpy array}) en cada iteraci\'on. 
\item \textbf{Proceso de estimaci\'on con filtro de Kalman:} En este paso se realizan los procesos de predicci\'on y correcci\'on para obtener una nueva estimaci\'on. Se usa alguna instancia de uno de las clases \textsc{KalmanFilter} o \textsc{MaximumCorrentropyKalmanFilter} (en el programa original s\'olo est\'an implementados los filtros b\'asico y de m\'axima correntrop\'ia). 
\item \textbf{Detecci\'on de fuentes:} En este paso se estudia la idoneidad de los p\'ixeles tanto del flujo como de las estimaciones de los mismos (obtenidos con el filtro de Kalman) y del resto de las im\'agenes al momento de verificar una serie de criterios con los cuales estos se agrupar\'ian entre vecinos para formar candidatos (este proceso se realiza en \textsc{SNDetector}) (estos criterios son heredados en el futuro por la versi\'on refactorizada de este programa, en la clase \textsc{SourceFinder}). En la Figura \ref{fig:example}
se ejemplifica este proceso de agrupamiento.

\begin{figure}%
    \centering
    \subfloat[Los p\'ixeles son evaluados individualmente.]{{\includegraphics[width=5cm]{images/source.png} }}%
    \qquad
    \subfloat[Fuentes encontradas a partir de agrupamiento.]{{\includegraphics[width=5cm]{images/ccd_field.png} }}%
    \caption{La figura de la izquierda, corresponde a una muestra de una imagen cient\'ifica, en donde se logra observar sus p\'ixeles los cuales deben ser evaluados individualmente primero. Luego, en la figura derecha, se destaca en un c\'irculo rojo, la supernova conocida (SN34) y en c\'irculos verdes las fuentes candidatas a supernova, como resultado del proceso de agrupamiento de los p\'ixeles que satisfacen cierto conjunto de criterios.}%
    \label{fig:example}%
\end{figure}
 
\item \textbf{Actualizaci\'on de candidatos:} Revisa si hay nuevos candidatos a ser considerados. Los nuevos candidatos se registran en un arreglo (\texttt{numpy array}), mientras que los guarda la informaci\'on de la supernova conocida, en caso de encontrarse.
\end{enumerate}

Si en el proceso anterior se encuentran candidatos, se procede a repetir los pasos de obtenci\'on de flujos, estimaciones y filtrado de p\'ixeles. La diferencia est\'a en que en esta ocasi\'on se van guardando la informaci\'on de estos. El nuevo ciclo queda como sigue:

\begin{enumerate}

\item \textbf{Repetici\'on de los pasos anteriores 1-3}
\item \textbf{Guardado de resultados:} La informaci\'on de los nuevos candidatos encontrados previamente es guardada en la lista diccionarios \textit{obj} (variable de la clase \textsc{Observer}) y registrado en disco usando el formato NPZ (formato brindado por \textsc{Numpy} para comprimir datos).
\end{enumerate}


El diagrama de la Figura \ref{fig:des_sif} entrega una perspectiva general de la secuencia de pasos que realiza el programa. Sin embargo, nos encontramos con el primer problema de implementaci\'on: la existencia de c\'odigo duplicado. Ya que ambos bloques est\'an repitiendo los mismos pasos salvo el \'ultimo en el que se diferencian en el registro de los resultados: en uno se guarda la informaci\'on de la supernova conocida (s\'olo si la detecta) y en el segundo, s\'olo si hay nuevos candidatos, guarda la informaci\'on de estos.
\bigskip

Cabe destacar que durante el proceso de estudio del programa original se encontr\'o que la lista de archivos, que debe ser procesada en orden cronol\'ogico, de acuerdo a la \'epoca (o fecha de observaci\'on), no est\'a siendo bien filtrada durante el proceso de selecci\'on de im\'agenes cient\'ificas en la clase \textsc{FITSHandler}, lo que puede involucrar imprecisiones en la detecci\'on de candidatos.
\bigskip


\begin{figure}[h!]
\centering
\includegraphics[scale=.5]{/home/paloma/Documents/Memoria/Code/sif2/sif_act}
\caption{Diagrama de flujo del programa original. Se aprec\'ian dos ciclos principales: el primero est\'a destinado a la b\'usqueda de una supernova de HiTS, y el segundo a la revisi\'on de la lista de posibles candidatos encontrados en el primer bloque del proceso, para el guardado de la informaci\'on (extracci\'on de estampillas, desde cada matriz, centradas en sus coordenadas) de estos candidatos. Notar que hay pasos que se repiten en la realizaci\'on de ambos an\'alisis.}
\label{fig:des_sif}
\end{figure}

\section{Estructura de datos}
\label{des:struct}
Debido a la naturaleza de la informaci\'on de entrada (im\'agenes) se debe trabajar en p\'ixeles, por lo que la estructura de datos que representen las variables de estado debe considerar la dimensi\'on de las im\'agenes (teniendo en cuenta que por cada p\'ixel se debe modelar las variables de estado flujo y velocidad de flujo, y covarianza asociada). Debido a esto, en el trabajo de Pablo Huentelemu \cite{huentelemu} se dise\~naron \textit{hiperrect\'angulos} para modelar las variables de estado y covarianza de todos los p\'ixeles de una imagen. Los datasets contienen im\'agenes de dimensi\'on  $2046 \times 4094$. La Figura \ref{fig:data_scheme} muestra el esquema de sistema de matrices o \textit{hiperrect\'angulos} usada para representar el estado de cada p\'ixel y su respectiva covarianza.
\bigskip  

\begin{figure}
\centering
\includegraphics[scale=.35]{/home/paloma/Documents/Memoria/SVG/hola.png}
\caption{Esquema de las estructuras de datos usadas para la representaci\'on de las matrices de estados y de las matrices de covarianza de la relaci\'on entre el flujo ($x$) y la velocidad de flujo ($\dot{x}$) por cada p\'ixel, de una imagen de ancho $w$ y altura $h$. Para los conjuntos de datos, $w=2046$ y $h=4094$.}
\label{fig:data_scheme}
\end{figure}
\bigskip

\section{Pruebas}

Se realizaron dos tipos de prueba: la primera, orientada a medir tiempo de ejecuci\'on y uso de memoria principal, se llev\'o a cabo en un computador personal de 8 GB (DDR4) de RAM y una CPU de frecuencia 2.80 GHz sobre tres series de datos definidos por pares de CCD y campo en los que se sabe que hay una supernova. El segundo tipo de prueba se realiz\'o en Leftraru, pasando por algunas complicaciones relacionadas con el dise\~no del programa ya que el c\'odigo original contiene demasiadas l\'ineas con el comando \texttt{glob}\footnote{\url{https://docs.python.org/3.5/library/glob.html}} de Python (y una de las razones de porqu\'e se decidi\'o realizar el primer tipo de prueba en un computador personal), el cual lista reiteradamente el contenido de los directorios del \textit{home} del usuario lo que finalmente termina saturando el nodo destinado para el lanzamiento del programa si se tiene una gran cantidad de archivos. Este problema pudo resolverse eliminando documentos resultantes de pruebas iniciales. Cabe destacar que se adapt\'o el programa para Python 3.6, ya que originalmente estaba para 2.7 (ambos tipos de pruebas se realizaron para la versi\'on 3.6). En los dos tipos de prueba, se usaron los valores por defecto de los par\'ametros, estos son descritos en las secciones siguientes.
\bigskip


\subsection{Tiempo de ejecuci\'on}

El estudio del tiempo de ejecuci\'on del programa se realiz\'o usando la funci\'on \texttt{getrusage} de la librer\'ia \texttt{resource} de Python 3.6, midiendo el tiempo de usuario en segundos. Las mediciones se realizaron sobre tres conjuntos de datos, los cuales contienen alguna supernova detectada por HiTS, y que fueron seleccionados al azar: SN14, SN18 y  SN80. Cada uno de ellos comprende secuencias de 26, 23 y 18 observaciones, respectivamente. 
\bigskip

Las Tablas \ref{tab:t1} y \ref{tab:t2} muestran el tiempo en segundos que toma el proceso de detecci\'on de candidatos. Se destacan como procesos separados el reconocimiento de la supernova de HiTS y posteriormente, el proceso de estudio de los nuevos candidatos, respectivamente, empleando para ambos procesos el filtro b\'asico.  

\begin{table}[h!]
\centering
\caption{Resultados de tiempos de ejecuci\'on correspondientes a c\'alculo de flujo, estimaci\'on del filtro, agrupaci\'on de p\'ixeles y filtrado de los mismos durante el per\'iodo de reconocimiento de la supernova correspondiente. Para esta prueba se utiliz\'o el filtro de Kalman b\'asico. La \'ultima fila corresponde a la media por observaci\'on.}
\begin{tabular}{|l|r|r|r|r|}
\hline
\textbf{ID} & \textbf{C\'alc. Flujos [s]} & \textbf{Aplic. KF [s]} &  \textbf{Agrup. P\'ixeles [s]}  & \textbf{Actual. Candidatos [s]}\\ \hline \hline
SN14        & 293,91            & 24,90        &  68,30 & 0,02 \\ \hline
SN18            & 260,09             & 22,56         &  45,52  & 0,00\\ \hline
SN80            & 204,93             & 17,69         &   38,21 & 0,00 \\ \hline \hline
%Media & 303.08 &  26.23 & 37.83 & 0.01\\\hline 
$\bar{t}/Obs$ & 11,00 &  0,97 & 2,24 & 0,00\\\hline 
\end{tabular}
\label{tab:t1}
\end{table}

\begin{table}[h!]
\centering
\caption{Resultados de tiempos de ejecuci\'on correspondientes a c\'alculo de flujo, estimaci\'on de los filtros, agrupaci\'on de p\'ixeles y filtrado de los mismos durante el per\'iodo de estudio de los nuevos candidatos encontrados en el paso anterior. Para esta prueba se utiliz\'o el filtro de Kalman b\'asico. Se observa que para las dos \'ultimas supernovas los tiempos son cero ya que no se encontraron m\'as candidatos.}
\begin{tabular}{|l|r|r|r|r|}
\hline
\textbf{ID} & \textbf{C\'alc. Flujos [s]} & \textbf{Aplic. KF [s]} &  \textbf{Agrup. P\'ixeles [s]}  & \textbf{Guardar resultados [s]}\\ \hline \hline
SN14        & 303,18            & 27,28        &  72,34 & 0,02 \\ \hline
SN18            & 0,00             & 0,00         &  0,00  & 0,00\\ \hline
SN80            & 0,00             & 0,00         &   0.00 & 0,00 \\ \hline\hline 
%Media & 306.98 &  28.89 & 38.89  & 0.07\\\hline 
$\bar{t}/Obs$ & 11,66 &  1,05 & 2,78 & 0,00\\\hline 
\end{tabular}
\label{tab:t2}
\end{table}

Las Tablas \ref{tab:t3} y el \ref{tab:t4} describen el tiempo (en segundos) tomado tanto para el proceso de detecci\'on de la supernova de HiTS y el posterior procesamiento de posibles nuevos candidatos (en ese orden) usando el filtro de m\'axima correntrop\'ia.
\bigskip
 
\begin{table}[h!]
\centering
\caption{Resultados de tiempos de ejecuci\'on correspondientes a c\'alculo de flujo, estimaci\'on del filtro de Kalman, agrupaci\'on de p\'ixeles y filtrado de los mismos durante el per\'iodo de reconocimiento de la supernova correspondiente. Para esta prueba se utiliz\'o el filtro de Kalman de m\'axima correntrop\'ia.}
\begin{tabular}{|l|r|r|r|r|}
\hline
\textbf{ID} & \textbf{C\'alc. Flujos [s]} & \textbf{Aplic. KF [s]} &  \textbf{Agrup. P\'ixeles [s]}  & \textbf{Actual. Candidatos [s]}\\ \hline \hline
SN14        & 342,44            & 798,48        &  76,47 & 0,00 \\ \hline
SN18            & 273,64             & 566,89         &  47,96  & 0,00\\ \hline
SN80            & 210,68             & 420,12         &   36,05 & 0,00 \\ \hline \hline 
%Media & 309.32 & 638.48 &  37.07 & 0.01\\\hline
$\bar{t}/Obs. $& 12,25 & 26,23 & 2,34 & 0,00\\\hline 
\end{tabular}
\label{tab:t3}
\end{table}

\begin{table}[h!]
\centering
\caption{Resultados de tiempos de ejecuci\'on correspondientes a c\'alculo de flujo, estimaci\'on de los filtros, agrupaci\'on de p\'ixeles y filtrado de los mismos durante el per\'iodo de estudio de los nuevos candidatos encontrados en el paso anterior. Para esta prueba se utiliz\'o el filtro de Kalman de m\'axima correntrop\'ia. Se observa que para las dos \'ultimas supernovas los tiempos son cero ya que no se encontraron m\'as candidatos.}
\begin{tabular}{|l|r|r|r|r|}
\hline
\textbf{ID} & \textbf{C\'alc. Flujos [s]} & \textbf{Aplic. KF [s]} &  \textbf{Agrup. P\'ixeles [s]}  & \textbf{Guardar resultados [s]}\\ \hline \hline
SN14        & 307,64            & 680,65        &  65,67 & 0,02 \\ \hline
SN18            & 0,00             & 0,00         &  0,00  & 0,00\\ \hline
SN80            & 0,00             & 0,00         &   0,00 & 0,00 \\ \hline \hline
%Media & 306.71 & 634.09 &  36.34 & 0.05\\\hline
$\bar{t}/Obs. $& 11,83 & 26,18 & 2,53 & 0,00\\\hline  
\end{tabular}
\label{tab:t4}

\end{table}
\bigskip

La Tabla \ref{tab:t5} muestra el tiempo total comprendido por ambos subprocesos usando el filtro de Kalman b\'asico. Se destaca la diferencia del consumo de tiempo ante la ausencia y presencia de nuevos candidatos a supernova. Por otro lado, la Tabla \ref{tab:t6} muestran las mismas medidas de tiempo para el filtro de m\'axima correntrop\'ia. Se destaca el aumento considerable de tiempo al usar este \'ultimo filtro en relaci\'on al primero. 
  
\begin{table}[h!]
\centering
\caption{Tiempo de ejecuci\'on de los procesos de b\'usqueda de supernova de HiTS, revisi\'on de los candidatos encontrados y tiempo total comprendido por ambos procesos usando filtro de Kalman b\'asico. La \'ultima fila corresponde a tiempo total promedio por observaci\'on.}
\begin{tabular}{|l|r|r|r|}
\hline
\textbf{ID} & \textbf{B\'usqueda SN [s]} & \textbf{Revisi\'on candidatos[s]} & \textbf{Tiempo total [s]} \\ \hline
\hline
SN14 & 387,13 & 402,82 & 789,95 \\\hline
SN18 & 328,17 & 0,00 & 328,17\\\hline
SN80 & 260,83 & 0,00 & 260,83 \\\hline\hline
%Media & 367.15 & 374.83 & 741.98  \\\hline
 $\bar{t}/Obs. $& 14,55 & -- & --\\\hline 
\end{tabular}
\label{tab:t5}
\end{table}


\begin{table}[h!]
\centering
\caption{Tiempo de ejecuci\'on de los procesos de b\'usqueda de supernova de HiTS, revisi\'on de los candidatos encontrados y tiempo total comprendido por ambos procesos usando filtro de Kalman de m\'axima correntrop\'ia.}
\begin{tabular}{|l|r|r|r|}
\hline
\textbf{ID} & \textbf{B\'usqueda SN [s]} & \textbf{Revisi\'on candidatos [s]} & \textbf{Tiempo total [s]} \\ \hline
\hline
SN14 & 1217,39 & 1053,98 & 2068,80\\\hline
SN18 & 888,49 & 0,00 & 0,00\\\hline
SN80 & 666,85 & 0,00& 0,00\\\hline \hline
 $\bar{t}/Obs. $& 40,83 & -- & --\\\hline 
\end{tabular}
\label{tab:t6}
\end{table}

\subsection{Uso de memoria}

La memoria ocupada por el programa se midi\'o en t\'erminos de MiB (mebibytes) usando la librer\'ia \texttt{memory\_profiler}. Posteriormente las mediciones en la unidad previamente mencionada fueron pasadas a MB (megabytes) \footnote{$1MiB\simeq 1.049MB$ }.
\bigskip

La Figura \ref{fig:mem_kbf} muestra el comportamiento del uso de memoria principal (en t\'erminos de mebibytes) para los tres conjuntos de datos empleados, usando el filtro de Kalman b\'asico. Se distingue un uso m\'as intensivo (m\'aximo alcanzado) durante el proceso en que se usaron los datos de la SN14, la cual no s\'olo posee una secuencia de im\'agenes mayor (de 26) sino tambi\'en corresponde a aquella en donde se encontraron posibles nuevos candidatos. En la Tabla \ref{tab:mem1} se expone el m\'aximo consumo generado durante la ejecuci\'on de la pipeline empleando el filtro b\'asico.
\bigskip

\begin{figure}[h!]
\centering
\subfloat[Memoria ocupada en SN14]{\label{fig:kbf_14}{\includegraphics[width=0.5\textwidth]{images/results/sn14_00}}}\hfill
\subfloat[Memoria ocupada en SN18]{\label{fig:kbf_18}{\includegraphics[width=0.5\textwidth]{images/results/sn18_00}}}\vfill
\subfloat[Memoria ocupada en SN80]{\label{fig:kbf_80}{\includegraphics[width=0.5\textwidth]{images/results/sn80_00}}}
\caption{Comportamiento de la memoria (en mebibytes) durante la ejecuci\'on para los tres conjuntos de datos  de las supernovas SN14, SN18 y SN80, usando el filtro de Kalman b\'asico.}
\label{fig:mem_kbf}
\end{figure}

 
\begin{table}[h!]
\centering
\caption{Memoria principal (en unidades de MB) usada durante la ejecuci\'on del programa original con la versi\'on b\'asica del filtro de Kalman.}
\begin{tabular}{|l|r|}
\hline
\textbf{ID} & Memoria [MB]\\\hline\hline
SN14 & 2282,42\\\hline
SN18 & 2063,02\\\hline
SN80 & 2021,27\\\hline
\end{tabular}

\label{tab:mem1}
\end{table}


La Figura \ref{fig:mem_mcc} muestra el consumo de memoria (en mebibytes) para la ejecuci\'on del programa original para los tres conjuntos de datos, usando el filtro de m\'axima correntrop\'ia. Se destaca un comportamiento similar al obtenido usando el filtro b\'asico debido a  la detecci\'on de posibles candidatos con el conjunto de datos de la SN14. Por otro lado, se desprende un mayor consumo de parte de este filtro en relaci\'on a la versi\'on b\'asica. La Tabla \ref{tab:mem2} describe el consumo m\'aximo de memoria alcanzado con el uso del filtro de m\'axima correntrop\'ia.
\bigskip

\begin{figure}[h!]
\centering
\subfloat[Memoria ocupada en SN14]{\label{fig:mc_14}{\includegraphics[width=0.5\textwidth]{images/results/sn14_01}}}\hfill
\subfloat[Memoria ocupada en SN18]{\label{fig:mc_18}{\includegraphics[width=0.5\textwidth]{images/results/sn18_01}}}\vfill
\subfloat[Memoria ocupada en SN80]{\label{fig:mc_80}{\includegraphics[width=0.5\textwidth]{images/results/sn80_01}}}
\caption{Comportamiento de la memoria (en mebibytes) durante la ejecuci\'on para los tres conjuntos de datos de las supernovas SN14, SN18 y SN80. En los tres lanzamientos se us\'o el filtro de Kalman de  m\'axima correntrop\'ia.}
\label{fig:mem_mcc}
\end{figure}


\begin{table}[h!]
\centering
\caption{Memoria principal (en unidades de MB) empleada durante la ejecuci\'on del programa original usando filtro de Kalman de m\'axima correntrop\'ia.}
\begin{tabular}{|l|r|}
\hline
\textbf{ID} & Memoria [MB]\\\hline\hline
SN14 & 3353,13\\\hline
SN18 & 3194,96\\\hline
SN80 & 3209,67\\\hline
\end{tabular}
\label{tab:mem2}
\end{table}

\subsection{Detecci\'on}
Se realizaron las pruebas de detecci\'on sobre los conjuntos de datos de las 93 supernovas en Leftraru durante el mes de febrero de 2019, usando los  umbrales de 200.0 para estimaci\'on de flujo y 50.0 para la velocidad de flujo. La Tabla \ref{tab:tpfn}, muestra el n\'umero de detecciones exitosas como verdaderos positivos (TP) as\'o como detecciones fallidas en t\'erminos de falsos negativos (FN) y falsos positivos (FP), y escenarios en donde la ejecuci\'on del programa fall\'o (por no encontrar archivos necesarios). 
\bigskip

\begin{table}[h!]
\centering
\caption{N\'umero de verdaderos positivos (TP), falsos negativos (FN) y falsos positivos (FP) (objetos no considerados por HiTS como supernova) encontrados usando cada uno de los filtros. La cuarta columna, \textbf{NaN} indica el n\'umero conjunto de datos que no se pudieron procesar por falta de alg\'un(os) archivo(s). No se observan diferencias sustanciales entre los resultados de cada filtro. Con el filtro MCC se encontr\'o un falso positivo menos.}
\begin{tabular}{|l|r|r|r|r|}
\hline
\textbf{Filtro} & \textbf{TP} & \textbf{FN} & \textbf{FP}  & \textbf{NaN}\\ \hline
Básico          & 33          & 57         & 53 & 3 \\ \hline
MCC             & 33          & 57         & 52 & 3 \\ \hline
\end{tabular}
\label{tab:tpfn}
\end{table}

Para los par\'ametros usados (p\'arrafo anterior) se obtienen resultados id\'enticos para ambos filtros en t\'erminos de cantidad de supernovas detectadas. En ambos casos se detectaron 33, sin embargo existe una leve diferencia en la cantidad de falsos positivos: el programa, usando el filtro de m\'axima correntrop\'ia encuentra un falso positivo menos que al usar el filtro b\'asico. Para ambos casos, tres secuencias de supernovas no pudieron ser procesadas (cuyos resultados fueron etiquetados como \textit{NaN}) debido a la falta de im\'agenes de invarianza inversa (esenciales para el c\'alculo de flujo). Cabe destacar que para estos experimentos no se us\'o la optimizaci\'on de Silverman, ya que su implementaci\'on estaba incompleta.  %, ya que para tres supernovas (de las 93 y \'ultimas en la lista) no se encontraron las im\'agenes de invarianza inversa.
\bigskip

%Se debe se\~nalar que durante las primeras pruebas con el programa original, en su versi\'on para Python 2.7, se detectaban 33 supernovas de las 93, no obstante al actualizar el c\'odigo para una versi\'on compatible con Python 3.6, se obtiene un mejor resultado para los dos tipos de filtros, detect\'andose dos supernovas m\'as. Se piensa que esto puede deberse a la actualizaci\'on de la librer\'ia \textsc{PyMorph}\footnote{\url{https://pythonhosted.org/pymorph/}} por \textsc{Mahotas}\footnote{\url{https://mahotas.readthedocs.io/en/latest/}} (ambas librer\'ias desarrolladas por el mismo autor, Luis P. Coelho\footnote{\url{https://github.com/luispedro}}). 
%\bigskip

\subsection{Observaciones}
A partir de los resultados obtenidos, se procedi\'o a estudiar las curvas de luz (cuyo flujo se mide ADU) \footnote{analog-to-digital unit} considerando escenarios en donde supernovas conocidas son detectadas por el programa (con ambos filtros) (Figura \ref{fig:sns_found}) y donde no ocurre (Figura \ref{fig:sns_not_found}). 

\begin{figure}[h!]
\centering
\includegraphics[scale=0.5]{/home/paloma/Documents/Memoria/Background/light_curves/found0.png}
\caption{Curvas de luz (flujo en ADU vs tiempo en MJD) de 16 supernovas detectadas tanto por el filtro de Kalman b\'asico como el de m\'axima correntrop\'ia. Esta figura corresponde a un an\'alisis exploratorio de resultados obtenidos a trav\'es de m\'etodos fotom\'etricos tradicionales (no por filtros de Kalman), y est\'an disponibles junto a la base de datos de HiTS \cite{hits}.}
\label{fig:sns_found}
\end{figure}

\begin{figure}[h!]
\centering
\includegraphics[scale=0.5]{/home/paloma/Documents/Memoria/Background/light_curves/not_found0.png}
\caption{Curvas de luz (flujo en ADU vs tiempo en MJD) de 16 supernovas no detectadas ni por el filtro de Kalman b\'asico ni por el de de m\'axima correntrop\'ia. Esta figura corresponde a un an\'alisis exploratorio de resultados obtenidos a trav\'es de m\'etodos fotom\'etricos tradicionales (no por filtros de Kalman), y est\'an disponibles junto a la base de datos de HiTS \cite{hits}. Se aprecia el r\'apido crecimiento en la luminosidad de una de ellas.}
\label{fig:sns_not_found}
\end{figure}

En la Figura \ref{fig:sns_found} se observa, a grandes rasgos, la presencia de una etapa de crecimiento en el flujo de los objetos (supernovas), coincidiendo con el per\'iodo de alta cadencia mencionado en el cap\'itulo anterior (Cap\'itulo \ref{ch:background}, subsecci\'on \ref{ssec:data}), mientras que por el contrario en la imagen \ref{fig:sns_not_found} el comportamiento de los flujos es m\'as bien irregular y no se distinguen etapas de crecimiento constante. Hay que mencionar que el algoritmo principal, aquel cuya tarea es la de filtrar y agrupar p\'ixeles bajo criterios de descarte (implementado en \textsc{SNDetector}), emplea un sistema de \textit{alerta} interno con el cual se gatilla una detecci\'on cuando un grupo determinado de p\'ixeles satisface una serie de requisitos durante cuatro \'epocas consecutivas. Para gatillar esta alerta, uno de los requisitos que debe cumplir un grupo de p\'ixeles para considerarse candidato a supernova es el crecimiento continuo de los flujos por p\'ixel. Si ocurre un descenso entonces el grupo es inmediatamente descartado.  
%Al estudiar los periodos de las detecciones de cada uno de los conjuntos de datos de las supernovas de HiTS, para ambos filtros, se obtiene que no hay diferencias en la \'epoca en que se realiza (es decir misma hora o MJD). Ver Ap\'endice, secci\'on \ref{ap:tab1}. 
\bigskip

Las Figuras \ref{fig:orig_det_snL} y \ref{fig:orig_det_snaa} muestran dos casos del intervalo de tiempo en que se realizan detecciones exitosas usando ambas implementaciones de los filtros del programa original (ver Ap\'endice, secci\'on \ref{ap:tab1} para conocer lista completa de fechas de detecciones).
\bigskip

\begin{figure}[h!]
\centering
\includegraphics[scale=0.35]{/home/paloma/Documents/Memoria/SVG/hits_snL.png}
\caption{Curva de luz (en ADU vs MJD) de supernova registrada en el CCD N27, campo 34. La detecci\'on realizada por los filtros b\'asico y de m\'axima correntrop\'ia se realiz\'o en el MJD 57072.214 (\'epoca u observaci\'on 7 de 18). }
\label{fig:orig_det_snL}
\end{figure}


\begin{figure}[h!]
\centering
\includegraphics[scale=0.35]{/home/paloma/Documents/Memoria/SVG/hits_snaa.png}
\caption{Curva de luz supernova observada en el CCD S5, campo 21. Ambos filtros detectaron la supernova en el MJD 57077.166 (\'epoca u observaci\'on 23 de 27).}
\label{fig:orig_det_snaa}
\end{figure}
\bigskip

\subsubsection{Falsos positivos}
Para reconocer falsos positivos se requiere contar con una herramienta externa de visualizaci\'on o de \textit{machine learning}, por ejemplo, usando un clasificador de conjuntos de p\'ixeles que eval\'ue la probabilidad de que se trata efectivamente de una supernova joven. Esto se debe a que existen variados objetos como estrellas variables o \textit{artefactos} que corresponden a zonas ruidosas de la imagen  cient\'ifica (en particular, bordes) y pueden proveer valores alterados de flujo lo que en algunos casos puede conllevar a la detecci\'on de falsos positivos.
\bigskip

Una forma de poder estudiar un falso positivo es a trav\'es de la visualizaci\'on de su espacio de fase y la obtenci\'on de la entrop\'ia de la curva.
\chapter{Refactoring}
\label{ch:refactoring}
El refactoring del c\'odigo original se realiz\'o en Python 3.6. Adem\'as se mantuvieron casi la totalidad de las librer\'ias cambiando \'unicamente Pymorph por Mahotas, debido a que la primera dej\'o de ser mantenida desde el 2010 y la segunda corresponde a la librer\'ia para Python 3.   

\section{Manejo de la rutina: \textsc{RoutineHandler}}

La \textbf{rutina} comprende el proceso de listado de archivos (que en esta nueva versi\'on se realiza en la clase \textsc{DataPicker}); los procesos iterativos de c\'alculo de flujo, obtenci\'on de estimaciones y la detecci\'on de candidatos con \textsc{SourceFinder}; y el guardado de la informaci\'on de los  candidatos y/o la generaci\'on de visualizaciones (dependiendo del tipo de ejecuci\'on). Estos procesos son administrados por la clase \textsc{RoutineHandler} el cual debe instanciarse recibiendo tres archivos descritos a continuaci\'on:


\begin{itemize}
\item Lista de campos, CCDs, semestres y, alternativamente, las coordenadas de alg\'un objeto de inter\'es, en un archivo CSV (en este mismo orden), teniendo como encabezado \texttt{Field, CCD, Semester, POS\_Y, POS\_X}. En caso de no adjuntar las coordenadas en los campos de posici\'on, se puede rellenar con un gui\'on (`-'). En el constructor este archivo se recibe como \texttt{obs\_index\_path} (ejemplo en Ap\'endice \ref{subs:sn_list}).
\item Diccionario de directorios y expresiones regulares de las ubicaciones de los archivos y sus nombres, respectivamente (archivo de formato CSV). En el constructor de la clase este archivo se denomina como \texttt{routes\_templates} (ejemplo en Ap\'endice \ref{subs:a1}).
\item Diccionario de umbrales y par\'ametros relevantes en la ejecuci\'on del programa, as\'i como el tipo de filtro a usar (archivo de formato CSV). En el constructor se denomin\'o \texttt{settings\_file} (ver ejemplo en Ap\'endice \ref{subs:settings_file}).
\end{itemize}
\bigskip

Adem\'as de estos archivos se debe especificar el \'indice de la fila a procesar de la lista de campos, CCDs y semestre (primer archivo de entrada). Esto se hizo as\'i con la finalidad de facilitar la paralelizaci\'on de los an\'alisis de diferentes conjuntos de datos. Por lo tanto el constructor de \textsc{RoutineHandler} queda como sigue:

\begin{center}
\texttt{RoutineHandler(obs\_index\_path, routes\_templates, settings\_file, index)}
\end{center}

El m\'etodo m\'as importante de esta clase es \texttt{routine}, el cual implementa la pipeline principal del programa orignal. Este m\'etodo puede ser ejecutado de dos formas, dependiendo de la finalidad que busca el usuario: si se desea realizar una b\'usqueda de candidatos (considerando tambi\'en las coordenadas entregadas en \texttt{obs\_index\_path}) o, si se desea visualizar informaci\'on de candidatos ya encontrados (a partir de una lista de coordenadas cargadas desde un archivo NPZ). La Figura \ref{fig:new_routine} muestra ambas formas de ejecuci\'on. La lista completa de m\'etodos de esta clase se encuentran en el Ap\'endice, secci\'on \ref{subs:a4}.

\begin{figure}
\centering
    \subfloat[B\'usqueda de candidatos]{{\includegraphics[scale=.5]{/home/paloma/Documents/Memoria/Code/sif2/sif2_act} }}%
    \qquad
    \subfloat[Generaci\'on de visualizaciones]{{\includegraphics[scale=.5]{/home/paloma/Documents/Memoria/Code/sif2/sif2_act2} }}%
\caption{Rutina del programa refactorizado. Cuenta con dos modos de ejecuci\'on: uno de b\'usqueda de candidatos (izquierda) y otro de generaci\'on de gr\'aficos (derecha). Pueden ejecutarse ambas formas comenzando con la b\'usqueda de candidatos y continuar con el modo de generaci\'on de gr\'aficos. El usuario tiene la libertad de escoger el modo de ejecuci\'on.}
\label{fig:new_routine}
\end{figure}  
%\section{Ejecuci\'on de la rutina}
\section{Manejo de datos de entrada}
Todas las im\'agenes de entrada son manipuladas y servidas por la clase \textsc{DataPicker}. Esta clase se inicializa recibiendo un \textit{path} hacia el archivo de configuraci\'on \texttt{routes\_templates} que contiene tanto las rutas a los archivos as\'i como los nombres de estos en t\'erminos de expresiones regulares; semestre a los que corresponde la secuencia de observaciones (los dos \'ultimos d\'igitos del a\~no concatenados con la letra A en caso de corresponder al primer semestre o B al segundo); el campo (representado como un n\'umero de dos d\'igitos, comenzando con cero para valores menores a 10) y el detector CCD (cadena de tres car\'acteres donde el primero de ellos describe a que grupo de detectores corresponde: N o S (ver Figura \ref{fig:f4}); adem\'as de un n\'umero entero que va de 1 a 36) como strings (para m\'as detalles, ver Ap\'endice, secci\'on \ref{subs:des_rutas}). 
\bigskip

Es en \textsc{DataPicker} donde se realiza el proceso de filtrado de las im\'agenes (descarte por \gls{airmass}) y su posterior ordenamiento, esta vez considerando un filtrado m\'as minucioso, considerando la correspondencia de las fechas MJD contenidas en los header de cada archivo FITS. Para ver los detalles de este proceso de listado de im\'agenes y filtrado, revisar el Ap\'endice, secci\'on \ref{subs:a2}.

\section{Determinaci\'on de flujos}
El c\'alculo del flujo, en este refactoring, se independiz\'o del manejo de archivos (en el programa original estaba alojado en la clase \textsc{FITSHandler}) y se implement\'o en el script \texttt{utils} pensado como librer\'ia (una descripci\'on detallada de los m\'etodos se encuentra en el Ap\'endice, secci\'on \ref{ap:utils}).
\bigskip

\section{Filtros originales}
La refactorizaci\'on de los filtros de Kalman originales implic\'o la implementaci\'on de nuevas clases e interfaces para el desarrollo del patr\'on propuesto: Strategy (ver familia de m\'etodos resultante en la Figura \ref{fig:ref1}). A continuaci\'on se presentan cada una de ellas:

\begin{itemize}
\item \textbf{IPredict:} Interface que describe el comportamiento de la funci\'on \textsc{Predict} de un filtro de Kalman. El m\'etodo \texttt{predict} recibe como par\'ametros el paso de tiempo ($\Delta t$), la matriz de estado, la matriz de covarianza de estado, y las predicciones de las matrices de estado y covarianza determinadas en el paso anterior (con la finalidad de actualizar estas variables). Su firma queda como:
\begin{center}
\texttt{predict(delta\_t, state, state\_cov, pred\_state, pred\_cov)}
\end{center}
Este m\'etodo entrega finalmente las matrices de estado y covarianza de estado predicho.
\bigskip

\item \textbf{ICorrect:} Interface que describe el comportamiento de la funci\'on \textsc{Correct} de un filtro de Kalman. El m\'etodo \texttt{correct} recibe como par\'ametros la matriz de flujo (\texttt{z}, correspondiente a la observaci\'on) y de varianza de la observaci\'on (\texttt{R}), las matrices de estado y covarianza predichas y las matrices de estado y covarianza estimadas obtenidas en el paso anterior y que ser\'an actualizadas. Su firma queda como:
\begin{center}
\texttt{correct(z, R, pred\_state, pred\_cov, state, state\_cov)}
\end{center}
\bigskip
Esta funci\'on entrega finalmente las matrices de estado y covarianza de estado corregido.

\end{itemize}

\subsection{Predicci\'on}
\textbf{LinearPredict:} Clase que extiende IPredict. Implementa el m\'etodo \texttt{predict} que ser\'a usado tanto por el filtro b\'asico como por el de m\'axima correntrop\'ia. Su instanciaci\'on recibe como argumento \texttt{sigma\_a} (desviaci\'on est\'andar del ruido del proceso, asumiendo una distribui\'on gaussiana para las observaciones).
\bigskip


\subsection{Correcci\'on}
\textbf{BasicCorrect:} Clase que extiende ICorrect. Implementa el m\'etodo \texttt{correct} que ser\'a usado para el tipo de filtro de Kalman b\'asico.
\bigskip

\textbf{MCCorrect:} Clase que extiende ICorrect. Implementa el m\'etodo \texttt{correct} que ser\'a usado para el tipo de filtro de Kalman de m\'axima correntrop\'ia. El constructor de esta clase recibe los siguientes par\'ametros:

\begin{itemize}
\item \texttt{epsilon}: Cantidad con la cual se contrastar\'a el error o precisi\'on que se quiera lograr con la estimaci\'on (el valor definido por defecto en el programa original es de $10^{-6}$).
\item \texttt{max\_iter}: N\'umero m\'aximo de iteraciones (de acuerdo al c\'odigo orignal, se estableci\'o como $10$ su valor por defecto).
\item \texttt{silverman}: \textit{boolean}. Determina si se usa o no la apoximaci\'on de Silverman para el ancho de banda del kernel Gaussiano (por defect es \textit{false}).
\item \texttt{std\_factor}: Factor de desviaci\'on est\'andar usado en la aproximaci\'on de Silverman (su valor predeterminado es de $100,0$).
\item \texttt{sigma}: Sigma usado para el ancho de banda en el c\'alculo de la correntrop\'ia. Puede ser o no optimizado por Silverman. Por defecto, es $1000,0$, seg\'un programa original.
\end{itemize}

\subsection{Filtros refactorizados}

\begin{itemize}
\item \textbf{KalmanFilter:} Clase abstracta padre de los subtipos BasicKalmanFilter y MCKalmanFilter. Posee los m\'etodos abstractos \texttt{predict} y \texttt{correct}, que son definidos de acuerdo a las estrategias de predicci\'on y correci\'on descritas previamente.
\item \textbf{BasicKalmanFilter:} Representa el filtro b\'asico de Kalman. Est\'a compuesto por las estrategias \texttt{LinearPredict} y \texttt{BasicCorrect}.
\item \textbf{MCKalmanFilter:} Representa el filtro de m\'axima correntrop\'ia. Est\'a compuesto por las estrategias \texttt{LinearPredict} y \texttt{MCCorrect}.
\end{itemize}
\bigskip

\begin{figure}
\centering
\includegraphics[scale=.5]{images/kalmanfilter_class}
\caption{Familia de filtros de Kalman y patr\'on \textit{strategy} usado en la implementaci\'on de los m\'etodos \texttt{predict} y \texttt{correct}.}
\label{fig:ref1}
\end{figure}

\section{Detecci\'on de candidatos}
\label{sec:detec}
Mientras que la detecci\'on de candidatos en el programa original se realiza en una instancia de la clase \textsc{SNDetector}, la detecci\'on en la nueva versi\'on se realiza en \textsc{SourceFinder}, el cual funciona de la misma forma que su antecesor. El cambio de nombre representa la intencionalidad de extender la funcionalidad de esta clase no s\'olo a la detecci\'on de supernovas sino adem\'as encontrar objetos como estrellas variables.
\bigskip

Su constructor requiere los siguientes argumentos:
\begin{itemize}
\item \texttt{flux\_thresh:} Umbral de corte para el flujo estimado por el filtro de Kalman.
\item \texttt{flux\_rate\_thresh:} Umbral de corte de velocidad de flujo estimado por el filtro de Kalman.
\item \texttt{rate\_satu:} Tasa de saturaci\'on en la velocidad de flujo.
\item \texttt{accum\_neg\_flux\_depth:} Cantidad de \'epocas de registro de p\'ixeles negativos (para la construcci\'on de una matriz de \textit{booleans} de esta profundidad).
\item \texttt{accum\_med\_flux\_depth:} Cantidad de \'epocas de registro de p\'ixeles cuya intensidad mediana (durante las \'epocas) es mayor a 1500.
\item \texttt{image\_size:} Dimensi\'on de las im\'agenes FITS (por defecto, corresponde a la tupla (4094, 2046)).
\item \texttt{n\_consecutive\_obs:} N\'umero de alertas u observaciones consecutivas a considerar para confirmar una detecci\'on. Para las versiones antigua y nueva de este programa, se estableci\'o como $4$.

\end{itemize}

Los m\'etodos se encuentran descritos en el Ap\'endice, secci\'on \ref{ap:sourcefinder}. Adem\'as, cabe mencionar que los par\'ametros previamente mencionados se ajustan al archivo de configuraci\'on, \texttt{settings\_file}.


\section{Visualizaci\'on de resultados}
En el proceso de visualizaci\'on participan dos clases: \textsc{Observer} y \textsc{Visualizer}. La primera clase es la encargada de generar una lista de diccionarios dentro de los cuales se almacena informaci\'on de los candidatos encontrados durante el proceso de detecci\'on. Esta lista es guardada como una variable de instancia en \textsc{Observer} denominada \texttt{objects} y la informaci\'on contenida por cada uno de los diccionarios corresponde a las siguientes componentes:

\begin{itemize}
\item Ubicaci\'on del objeto en la primera imagen cient\'ifica procesada, como un arreglo de \textit{floats} de largo dos. 
\item Lista de \'epocas en las que el objeto fue detectado.
\item Listas de estampillas de diferente profundidad, de ancho y alto $21 \times 21$. Las estampillas de cada lista tendr\'an una profundidad propia dependiendo desde la estructura que se extraigan. Estas estructuras pueden ser im\'agenes como la cient\'ifica, de diferencia, m\'ascara, etc., o a estructuras como las etiquetas de p\'ixeles grupales e individuales, matrices de estado, etc. Cada una de las estampillas de cada lista corresponder\'a a la medici\'on, c\'alculo o estimaci\'on obtenida para una \'epoca espec\'ifica. Adem\'as cada una de las estampillas est\'a centrada en la coordenada del objeto candidato a revisar.   
\end{itemize}
\bigskip

Para el registro de estos datos se emplean dos m\'etodos de \textsc{Observer} listados a continuaci\'on:

\begin{itemize}
\item \texttt{set\_space (cand\_data):}\\
Recibe lista de candidatos (coordenadas) \texttt{cand\_data} y crea lista de diccionarios, \texttt{objects}, con la finalidad de guardar la informaci\'on de los primeros para todas las \'epocas analizadas creando arreglos (de la librer\'ia \textsc{Numpy}) de estampillas $21 \times 21$ destinadas a guardar los datos de las diferentes estructuras: im\'agenes, matrices de estado estimado y predicho, covarianzas por pixel, matrices de etiquetas de p\'ixeles, etc. Cada estampilla estar\'a centrada en la posici\'on del candidato en la primera imagen cient\'ifica analizada. 
\bigskip

\item \texttt{look\_candData(cand\_data, pred\_state, pred\_state\_cov, kalman\_gain, state, state\_cov, time\_mjd, flux, var\_flux, science, diff, psf, base\_mask, dil\_base\_mask, pixel\_flags, pixel\_group\_flags, mjd\_idx):}\\
Con el espacio generado en la estructura \texttt{objects}, se procede a ejecutar la pipeline principal (c\'alculo de flujo, generaci\'on de estimaciones con filtro de Kalman y proceso de detecci\'on) para, en esta ocasi\'on, ir guardando resultados de los candidatos en \texttt{cand\_data} en su diccionario respectivo por cada \'epoca (cuyo \'indice est\'a representado por el argumento \texttt{mjd\_idx}). Por tanto recibe como argumentos las matrices de estados y covarianzas predichos (\texttt{pred\_state} y \texttt{pred\_state\_cov} respectivamente), las matrices de estados y covarianzas estimados (\texttt{state} y \texttt{state\_cov}) la matriz de ganancia de Kalman (\texttt{kalman\_gain}), matriz de flujo calculado y varianza asociada (\texttt{flux} y \texttt{var\_flux}), imagen cient\'ifica (\texttt{science}) y de diferencia (\texttt{diff}), imagen de la \gls{psf} usada (\texttt{psf}), matrices de etiquetas de p\'ixeles (\texttt{pixel\_flags} y \texttt{pixel\_group\_flags}) e imagen de la m\'ascara usada (\texttt{base\_mask}) junto a su versi\'on post-proceso de dilataci\'on (\texttt{dil\_base\_mask}).


\item \texttt{plot\_results(semester, field, ccd, plots\_path)}\\
Finalmente, con la variable de instancia \texttt{objects} generada, es posible crear los gr\'aficos de cada candidato con este m\'etodo, el cual recibe como par\'ametros el semestre (\texttt{semester}), el campo (\texttt{field}) y el CCD (\texttt{ccd}) de donde se obtuvieron las im\'agenes junto a la ruta al directorio donde se quiere guardar las im\'agenes (\texttt{plots\_path}) como argumentos. Los gr\'aficos son generados por una instancia de la clase \textsc{Visualizer}.
\end{itemize}

 
La clase \textsc{Visualizer} permite la obtenci\'on de tres tipos de gr\'aficas de acuerdo al m\'etodo llamado.
 
\begin{itemize}
\item \textbf{Curvas de luz, estados, etiquetas y varianzas}\\
Esta gr\'afica contiene cuatro series de tiempo cuyos datos se obtienen a partir del pixel ubicado justo en el centro del objeto considerado candidato. La primera serie de tiempo corresponde al flujo estimado y predicho por Kalman junto al flujo observado o medido. La segunda serie de tiempo corresponde a la componente de velocidad de flujo estimada y predicha de los estados obtenidos por Kalman. La tercera serie de tiempo corresponde a la evoluci\'on de las etiquetas (criterios que no satisface el pixel). Finalmente, la cuarta serie de tiempo, muestra la evolución de las varianzas de cada elemento de estado estimado y predicho, además del correspondiente a la medición.
  
\item \textbf{Estampillas}\\ 
Muestra el comportamiento de los p\'ixeles en las estampillas de dimensi\'on $21 \times 21$ de las siguientes estructuras: imagen cient\'ifica, \gls{psf}, flujos observado y su varianza, estados de flujo y su velocidad estimados por filtro de Kalman, etiquetas grupales e individuales de p\'ixeles.
\item \textbf{Curva de estado}\\
 Esta curva se logra a partir de los valores estimados de flujo y de la velocidad de esta obtenidos por el filtro de Kalman. Esta gr\'afica muestra la complejidad de la curva visualizada calculando su entrop\'ia \cite{balestrino}.
\end{itemize}

Estas gr\'aficas son generadas gracias a los siguientes m\'etodos de \textsc{Visualizer}:

\begin{itemize}
\item \texttt{print\_lightcurve(obj, obs\_rad, height, width, save\_filename):}\\
Este m\'etodo est\'a destinado a crear series de tiempo, para contrastar, de diferentes variables de inter\'es tales como el flujo observado, estimado y predicho (medidos en el pixel central del candidato) visualizados en la misma gr\'afica. Del mismo modo, en un gr\'afico dispuesto bajo el primero, se dibujan las series de tiempo de las velocidades estimadas y predichas. Posteriormente se grafica la evoluci\'on de las etiquetas de p\'ixeles individuales y grupales, y el \'ultimo gr\'afico generado corresponde a la serie de tiempo de las varianzas y covarianzas de las diferentes variables como flujo, estimaciones de flujo y sus predicciones obtenidas con el filtro de Kalman, etc.
\bigskip

 Los par\'ametros \texttt{height} y \texttt{width} corresponden a la altura y ancho de la imagen generada. Por \'ultimo, \texttt{save\_filename} corresponde al nombre con el que se guardar\'a el archivo en disco en formato PNG. Ejemplo de imagen resultante en Figura \ref{fig:lc_result}.
%\bigskip
 
\item \texttt{print\_stamps(obj, height, width, save\_filename):}\\
Recibe como entrada un diccionario de la lista de objetos almacenados por una instancia de \textsc{Observer}. Con esta funci\'on las estampillas son impresas secuencialmente (orden cronol\'ogico) en filas, donde cada una de estas \'ultimas corresponde a alg\'un tipo de imagen o estructura como la imagen cient\'ifica, estimaci\'on de flujo, etiquetas de p\'ixeles, etc. (ver ejemplo de imagen en la Figura \ref{fig:stamps_result}).
\bigskip

Los valores \texttt{height} y \texttt{width} corresponden a la dimensi\'on de la imagen resultante. El argumento \texttt{save\_filename} indica el nombre con el que se quiere guardar el documento (de formato PNG) en disco. 

\item \texttt{print\_space\_states(obj, obs\_rad, height, width, flux\_thresh, rate\_flux\_thresh,}\\
\texttt{save\_filename):}\\
Esta funci\'on es la encargada de graficar la curva de estados por los que pasa un candidato (cuya informaci\'on est\'a contenida en el diccionario \texttt{objects}). Se visualizan los estados en su detecci\'on y tres \'epocas anteriores previas a su confirmaci\'on (durantes las alertas). Estos estados est\'an definidos por el flujo y la velocidad de flujo estimados por el filtro de Kalman. Adem\'as en la misma imagen, se indica el nivel de complejidad de la curva en t\'erminos de entrop\'ia \cite{balestrino}.
\bigskip

El c\'aculo de la complejidad de la curva (o entrop\'ia de la curva) se obtiene a partir de la siguiente expresi\'on:

\begin{equation}
H(\Gamma) = log\left( \dfrac{2L}{C} \right), 
\label{eq:entropy_curve}
\end{equation}
donde $L$ corresponde al largo de la curva $\Gamma$ y $C$ es el largo de su envolvente convexa (\textit{convex hull})\cite{balestrino}.  Un ejemplo de imagen resultante se observa en la Figura \ref{fig:sp_st}, en donde el valor de la entrop\'ia de la curva se muestra en la leyenda.

Los argumentos del m\'etodo, \texttt{height} y \texttt{width}, corresponden a las dimensiones de alto y ancho de la imagen a generar, mientras que \texttt{save\_filename} indica el nombre del archivo generado a guardar.  
\end{itemize}

\begin{figure}[h!]
\centering
\includegraphics[scale=.45]{/home/paloma/Documents/PLOTS/BASIC/lc_sem_15A_field_38_ccd_S25_obj_1.png}
\caption{Conjunto de series de tiempo de diferentes componentes de inter\'es en el pixel ubicado en la coordenada de posici\'on del candidato. El primer gr\'afico (de arriba hacia abajo) muestra la evoluci\'on del flujo medido en contraste con el comportamiento de la predicci\'on y estimaci\'on realizada por el filtro de Kalman del mismo flujo durante las \'epocas de las observaciones. La segunda gr\'afica muestra los cambios de la predicci\'on y estimaci\'on de la velocidad de flujo (obtenidas por el filtro de Kalman) en el tiempo. El tercer esquema muestra la evoluci\'on de las etiquetas (grupal e individual) del pixel ubicado en la coordenada del candidato. Finalmente, el \'ultimo gr\'afico visualiza el comportamiento de las diferentes varianzas y covarianzas tanto de las componentes predichas y estimadas por el filtro (flujo y velocidad de flujo), as\'i como de las mediciones del mismo flujo (observado).}
\label{fig:lc_result}
\end{figure}

\begin{figure}[h!]
\centering
\includegraphics[scale=.45]{/home/paloma/Documents/PLOTS/BASIC/stamps_sem_15A_field_38_ccd_S25_obj_0.png}
\caption{Estampillas de matrices de $21 \times 21$ p\'ixeles y etiquetas que describen su comportamiento a trav\'es del tiempo: la primera fila de im\'agenes corresponde a estampillas obtenidas desde las im\'agenes cient\'ificas en donde debiese habitar la supernova observada durante todas las observaciones. La segunda fila muestra los diferentes modelos de PSF obtenidos para diferentes \'epocas. La tercera fila muestra el flujo observado en la misma posici\'on. Le sigue la varianza de este flujo. Posteriormente viene el flujo estimado por el filtro de Kalman siguiendo la velocidad de flujo estimado. Luego vienen las etiquetas de los p\'ixeles reconocidos por el programa como pertenecientes a un objeto transitorio (etiquetado por pixel y por grupo de p\'ixeles). La \'ultima fila corresponde a la m\'ascara base usada durante el an\'alisis.}
\label{fig:stamps_result}
\end{figure}
\bigskip


\begin{figure}[h!]
\centering

\includegraphics[scale=.5]{/home/paloma/Documents/PLOTS/BASIC/space_states_sem_15A_field_38_ccd_S25_obj_1.png}
\caption{Espacio de fase de flujo y velocidad de flujo de un candidato. En azul se destaca la estimaci\'on lograda por filtro de Kalman y en rojo el flujo observado versus la misma estimaci\'on de velocidad. Notar que en la leyenda de la figura, se indica el nivel de complejidad de la curva estimada en t\'erminos de su entrop\'ia \cite{balestrino} para la curva de flujo y velocidad de flujo estimado.}
\label{fig:sp_st}
\end{figure}
%El estilo de las gr\'aficas de la versi\'on refactorizada respet\'o el dise\~no original de las gr\'aficas, especificados por Pablo Huentelemu.
\chapter{Nueva funcionalidad}
\label{ch:news}
Se detallan en las siguientes secciones la nueva funcionalidad agregada (no incluye el manejo de los datos en \textsc{DataPicker} ya que se describe como parte del refactoring), con la que se pretende implementar la funcionalidad en l\'inea de  este programa. 
\section{Manejo de la rutina: \textsc{RoutineHandler}}
La \textbf{rutina} se entiende como la ejecuci\'on completa del programa, con la cual se procesan todas las observaciones de acuerdo a una lista de CCDs y campos. Para esta finalidad se cre\'o una clase llamada \textsc{RoutineHandler} la cual maneja los archivos de entrada de:
\begin{itemize}
\item Lista de campos, CCD y semestre en un archivo CSV (en este mismo orden) y con el encabezado \texttt{Field, CCD, Semester}.  
\item Diccionario de directorios y expresiones regulares de las ubicaciones de los archivos y sus nombres, respectivamente (archivo TXT). 
\item Diccionario de umbrales y valores relevantes en la ejecuci\'on del programa, as\'i como el tipo de filtro a usar (archivo TXT).
\end{itemize}
\bigskip

Esta clase contiene los siguientes m\'etodos:
\begin{itemize}
\item \texttt{process\_settings():}\\
En este m\'etodo se lee el archivo de diccionario de umbrales y valores con los que se configurar\'a la ejecuci\'on completa del programa.
\bigskip

\item \texttt{retrieve\_kalman\_filter(kalman\_string):}\\
Corresponde a un m\'etodo auxiliar que es invocado desde \texttt{process\_settings} con el que se crea una instancia del filtro de Kalman a partir de la lectura del archivo de valores, de acuerdo al valor defindo por el usuario. Los tres strings v\'alidos para la construcci\'on de una instancia son: 'Basic', 'MCC' y 'UKF'. Si se entrega otro tipo de string, se levanta un error.
\bigskip
  
\item \texttt{iterate\_over\_sequences():}\\
Recorre la lista de campos, CCDs y semestres entregada al programa con la consiguiente llamada a \texttt{routine}
\bigskip

\item \texttt{routine(semester, field, ccd, results\_path, last\_mjd):}\\
Corresponde a la rutina que comprende el an\'alisis de las observaciones de un semestre, campo y CCD espec\'ifico. 
\end{itemize}  
\bigskip

\section{Detecci\'on de fen\'omenos trasientes: \textbf{TPDetector}}
Es una peque\~na clase con la que se apoya el proceso de reconocimiento o detecci\'on de alg\'un fen\'omeno trasiente en el comportamiento de la intensidad de los pixeles calculados despu\'es la ejecuci\'on de \texttt{routine} (es decir, una vez que se han obtenido los resultados con \texttt{routine} de \textsc{RoutineHandler}).
\bigskip

Esta clase posee los siguientes m\'etodos:

\begin{itemize}
\item \texttt{look\_candidates(results\_path, field, ccd, semester)}:\\
Con este m\'etodo se agrupan los resultados obtenidos por campo, CCD y semestre en la ruta de los resultados (\texttt{results\_path}), y carga los arreglos de los candidatos encontrados (de acuerdo a su coordenada central) y cuenta las veces que aparece cada uno en los archivos (un archivo por \'epoca u observaci\'on) con la finalidad de registrar \textit{las veces que ha sido candidato}.  
\bigskip

\item \texttt{list\_candidates(cand\_mid\_coords)}\\
Con finalidades exploratorias, este m\'etodo registra los candidatos (por pares de pixeles que describen el centro de los grupos encontrados) sin repetici\'on, independiente de las veces que han aparecido.
\end{itemize}

  
\section{\textsc{DataContent}}
\textsc{DataContent} es una clase auxiliar que es usada principalmente para encapsular los resultados obtenidos durante el proceso de detecci\'on, tanto de los pixeles de los grupos encontrados por \'epoca, la lista de pixeles centrales de estos grupos, las matrices de etiquetas (para realizar seguimiento de las causas del descarte de los pixeles o grupo de pixeles), as\'i como tambi\'en los estados encontrados por el filtro usado. Para guardar estos resultados se utiliza la funci\'on \texttt{savez} de la librer\'ia \textsc{Numpy}, los archivos guardados son registrados en un archivo de extensi\'on NPZ.
\bigskip

Para el guardado de los datos se implement\'o la funci\'on \texttt{save\_results} que recibe como par\'ametros la ruta donde se quiere almacenar los resultados, el campo de observaci\'on, el CCD, el MJD y la matriz de estados junto con la covarianza de estados. El resto de los arreglos son definidos en las funciones \texttt{set\_mid\_coords} y \texttt{group\_info}.
\chapter{Resultados}
\section{Desempe\~no}
\subsection{Filtros b\'asico y de m\'axima correntrop\'ia}
\subsection{Filtro unscented}
\section{Pruebas en Leftraru}
\subsection{Filtro b\'asico}
\subsection{Filtro de m\'axima correntrop\'ia}
\subsection{Filtro unscented}
\label{ch:resultados}
\chapter{Conclusiones}
\label{ch:conclusion}



\section{Trabajo futuro}
El trabajo propuesto en esta tesis describe tres diferentes versiones de un nuevo m\'etodo de detecci\'on de fen\'omenos trasientes en r\'egimen creciente, basados tres variantes del filtro de Kalman, destinado a la detecci\'on de supernovas en etapa temprana. Sin embargo este mismo m\'etodo puede ser aplicado en la detecci\'on de estrellas variables usado un filtro que distinga tendencias decrecientes en la luminosidad de estos objetos; por lo que una extensi\'on prometedora de este sistema podr\'ia dise\~narse  para reconocer alternancia en los reg\'imenes creciente y decreciente, haciendo uso de la estructura de clases dada por el patr\'on Strategy en el modelo del filtro.
\bigskip

Por otro lado queda tambi\'en pendiente, el estudiar el comportamiento de los resultados al variar los umbrales relacionados con la estimaci\'on realizada por el filtro de Kalman y que son usados en la fase de reconocimiento de candidatos (en la clase \textsc{SourceFinder}). [M\'etodo basado en machine learning para estudiar falsos positivos??] 


% Apéndices
% Adicionales
%\begin{additional} 
%\section{Capítulo Adicional que no es apéndice}
%\end{additional}

% Apéndices
\begin{appendix}

\section{Librer\'ias usadas para el refactoring}
\label{subs:a0}
Versi\'on de Python: 3.6
\begin{itemize}
\item \textbf{\texttt{pandas}: 0.24.4}
\item \textbf{\texttt{matplotlib}: 2.2.2}
\item \textbf{\texttt{numpy}: 1.13.3}
\item \textbf{\texttt{mahotas}: 1.4.4}
\item \textbf{\texttt{astropy}: 3.0.2}
\end{itemize}
\pagebreak

\section{Rutas a directorios y expresiones regulares de archivos}
\label{subs:des_rutas}
Los campos se describen a continuaci\'on:
\begin{itemize}
\item \textbf{\texttt{maskDir}}: Directorio donde se almacenan las im\'agenes m\'ascara (im\'agenes que identifican p\'ixeles que no deben ser considerados).
\item \textbf{\texttt{scienceDir}}: Directorio donde se almacenan las im\'agenes cient\'ificas (im\'agenes base ya preprocesadas).
\item \textbf{\texttt{diffDir}}: Directorio donde se almacenan las im\'agenes de diferencia (resta entre las im\'agenes base y cient\'ifica).
\item \textbf{\texttt{psfDir}}: Directorio donde se encuentran los modelos de psf usados para la determinaci\'on del flujo.
\item \textbf{\texttt{invDir}}: Directorio que guarda las im\'agenes correspondientes a la varianza inversa (\textit{peso} de cada pixel en t\'erminos de ruido: a menor peso, mayor ruido).
\item \textbf{\texttt{afluxDir}}: Directorio que contiene los archivos de extensi\'on \texttt{NPY} dentro de los cuales se guarda el valor de la variable \texttt{aflux}.
\item \textbf{\texttt{maskRegEx}}: Expresi\'on regular con la que es posible identificar el nombre de las im\'agenes m\'ascara en disco siguiendo el path \texttt{maskDir}.
\item \textbf{\texttt{scienceRegEx}}: Expresi\'on regular con la que es posible identificar el nombre de las im\'agenes cient\'ificas en disco siguiendo el path \texttt{scienceDir}.
\item \textbf{\texttt{diffRegEx}}: Expresi\'on regular con la que se identifican el nombre de las im\'agenes de diferencia en disco siguiendo el path \texttt{diffDir}.
\item \textbf{\texttt{invRegEx}}: Expresi\'on regular con la que es posible identificar el nombre de las im\'agenes de la varianza inversa siguiendo el path \texttt{invDir}.
\item \textbf{\texttt{afluxRegEx}}: Expresi\'on regular con la que se identifica el nombre de los archivos \textit{match} que contienen el valor de \texttt{aflux}. Estos archivos est\'an ubicados en el path \texttt{afluxDir}.
\item \textbf{\texttt{psfRegEx}}: Expresi\'on regular que describe el nombre de las im\'agenes que guardan el modelo de PSF en el directorio \texttt{psfDir}.
\end{itemize}

\subsection{Archivo de entrada: configuraci\'on de paths}
\label{subs:a1}
\VerbatimInput{/home/paloma/Documents/Memoria/Code/sif2/inputs/dirset_leftraru.txt}

\pagebreak
\section{Diccionario de par\'ametros y umbrales}
\label{subs:a3}

\begin{itemize}
\item \texttt{imgHeight}: Altura de las im\'agenes cient\'ificas en p\'ixeles. Esta dimensi\'on se propaga al resto de im\'agenes y matrices. Las im\'agenes usadas en este trabajo poseen una altura de 4094 p\'ixeles. 
\item \texttt{imgWidth}: Ancho de las im\'agenes cient\'ificas en p\'ixeles. Esta dimensi\'on se propaga al resto de im\'agenes y matrices, y en este trabajo las respectivas im\'agenes poseen un ancho de 2046.
\item \texttt{filter}: Tipo de filtro (\texttt{basic}, \texttt{mcc} o \texttt{ukf}).
\item \texttt{results}: Directorio de resultados (donde se guardan las coordenadas de los candidatos encontrados junto a lista de \'epocas en que fueron detectados) en formato NPZ.
\item \texttt{init\_var}: Varianza inicial que tendr\'an las matrices de covarianza durante el proceso de estimaci\'on con los filtros de Kalman. 
\item \texttt{flux\_thresh}: Umbral para estado de flujo obtenido con Kalman. Es un valor determinado por el usuario (en este trabajo se defini\'o como 200). 
\item \texttt{flux\_rate\_thresh}: Umbral para la velocidad de flujo obtenido con Kalman. Es un valor establecido por el usuario (en este trabajo se defini\'o como 50).
\item \texttt{rate\_satu}: Tasa de saturaci\'on en la velocidad de flujo. Para la realizaci\'on de este trabajo se fij\'o como 3000, a partir de pruebas en las que se buscaba minimizar los falsos positivos.
\item \texttt{sigma\_a}: Varianza de la distribuci\'on de la componente de control ($u_k$) asumiendo normalidad. Es importante al emplear los filtros b\'asico y unscented, ya que corresponde a la desviaci\'on est\'andar de la distribuci\'on normal de la aceleraci\'on originada por cambios no esperados en el modelo (lo que se puede interpretar como ``fuerzas externas'') \cite{ian}.
\item \texttt{epsilon}: Radio de error con que la estimaci\'on por filtro de Kalman de m\'axima correntrop\'ia disminuye la ganancia de Kalman. Corresponde a un criterio de detenci\'on y toma valores entre 0 y 1 (t\'ipicamente, $10^{-6}$)\cite{badong}.
\item \texttt{max\_iter}: N\'umero de iteraciones m\'aximo para el proceso de correcci\'on al usar Kalman de m\'axima correntrop\'ia. 
\item \texttt{silverman}: \textit{boolean}. Toma valor 1 (\textit{true}) en caso de considerarse, y 0 (\textit{false}) si no. Se establece si se usa o no la aproximaci\'on de Silverman para determinar ancho de banda del kernel al emplear el filtro de m\'axima correntrop\'ia.
\item \texttt{std\_factor}: Factor de incremento de sigma al usar el m\'etodo de Silverman.
\item \texttt{sigma}: Ancho de banda usado por defecto sin Silverman en la determinaci\'on del kernel durante el proceso de correcci\'on con el Filtro de Kalman de correntrop\'ia m\'axima .
\item \texttt{beta}: Par\'ametro relacionado con la distribuci\'on del estado estimado ($x_k$). Toma valor de $\beta = 2$ para distribuciones normales.
\item \texttt{kappa}: Participa en la regulaci\'on del rango de los valores de los puntos sigma y usualmente toma valores entre 1 y 3N-1 (N corresponde al n\'umero de dimensiones)\cite{wan}. Aporta incremento adicional (ver Ecuaci\'on \ref{eq:eq21}).
\item \texttt{alpha}: Participa en la regulaci\'on del rango de los valores de los puntos sigma, y generalmente toma valores positivos menores o iguales a 1: $10^{-4} \leq \alpha \leq 1 $\cite{wan}. Incrementa el rango en un factor $\alpha$ (ver Ecuaci\'on \ref{eq:eq21}).
\item \texttt{dim}: Cantidad de componentes de estado a medir (en este programa se miden dos: flujo y su velocidad).
\end{itemize}
\subsection{Archivo de entrada: par\'ametro y umbrales}
\label{subs:settings_file}
\VerbatimInput{/home/paloma/Documents/Memoria/Code/sif2/inputs/settings_example.txt}
\pagebreak


\section{Archivo de entrada: lista de campos, CCDs y semestres (incluye algunas coordenadas)}
\label{subs:sn_list}
\VerbatimInput{/home/paloma/Documents/Memoria/Code/sif2/inputs/hits15.csv}
\pagebreak

\section{M\'etodos de la clase \textsc{RoutineHandler}}
\label{subs:a4}
\begin{itemize}
\item \texttt{process\_settings():}\\
En este m\'etodo se lee el archivo de diccionario de umbrales y par\'ametros con los que se configurar\'a la toma de decisiones del programa.
\bigskip

\item \texttt{retrieve\_kalman\_filter(kalman\_string):}\\
Corresponde a un m\'etodo auxiliar que es invocado desde \texttt{process\_settings} con el que se crea una instancia del filtro de Kalman a partir de la lectura del archivo de valores \texttt{settings\_file}. Los tres strings v\'alidos para la construcci\'on de una instancia son: `\texttt{basic}', `\texttt{mcc}' y `\texttt{ukf}'. Si se entrega otro tipo de string, se levanta un error.
\bigskip
  
\item \texttt{iterate\_over\_sequences(check\_found\_objects):}\\
Recorre la lista de campos, CCDs y semestres entregada al programa con la consiguiente llamada a \texttt{routine}. Recibe como par\'ametro el argumento \texttt{check\_found\_objects} con el cual se indica si se quiere analizar resultados obtenidos anteriormente (candidatos encontrados), y que es entregado al m\'etodo \texttt{routine} descrito a continuaci\'on.
\bigskip

\item \texttt{routine(semester, field, ccd, results\_path, check\_found\_objects, last\_mjd):}\\
Comprende la rutina principal del programa, es decir, el an\'alisis de las observaciones de un semestre, campo y CCD espec\'ifico. El argumento \texttt{check\_found\_objects} es un boolean e indicar\'a el modo de ejecuci\'on del m\'etodo: si es falso, s\'olo guardar\'a las coordenadas de los candidatos encontrados (si no encontr\'o nada, entonces se guarda una lista vac\'ia) adem\'as de las \'epocas en que fueron detectados. Esta informaci\'on se guarda en un arreglo de diccionarios. Si \texttt{check\_found\_objects} es verdadero, entonces cargar\'a resultados anteriores del directorio de resultados (configurado en \texttt{process\_settings}) para estudiar la presencia de los candidatos encontrados en caso de existir.
\end{itemize} 
\pagebreak

\section{M\'etodos de la clase \textsc{DataPicker}}
\label{subs:a2}

\begin{itemize}
\item \textbf{\texttt{config\_reg\_expressions(semester, field, ccd)}}\\
Este m\'etodo recibe como par\'ametros strings que indiquen el semestre (\texttt{semester}), el campo (\texttt{field}) y el ccd (\texttt{ccd}) que se quiere analizar. Puede hacerse uso de los valores de las variables de instancia que la misma clase \textsc{DataPicker} recibe en su constructor. Con estos strings se establecen las rutas de los directorios de las im\'agenes y las expresiones regulares de los nombres de las mismas.
\bigskip

\item \textbf{\texttt{collect\_data()}}\\
Esta funci\'on se encarga de recolectar la ruta completa de las diferentes im\'agenes (m\'ascaras, im\'agenes cient\'ificas, de diferencia, etc.). Para esta finalidad se hace uso del m\'etodo \texttt{walking\_through\_files}. 
\bigskip

\item \textbf{\texttt{walking\_through\_files(regex, dir)}}\\
M\'etodo con el cual se recorren las rutas definidas en los pasos anteriores y se agrupan los nombres completos (directorio incluido) de las im\'agenes ubicadas en el directorio \texttt{dir} y posean un nombre de patr\'on que siga la expresi\'on regular \texttt{regex}.
\bigskip

\item \textbf{\texttt{filter\_science\_images()}}\\
Filtra im\'agenes cient\'ificas de acuerdo a su \gls{airmass}, seleccionando aquellas obtenidas en fechas cuyo valor es menor a 1,7. De esta secuencia de im\'agenes cient\'ificas resultante se obtiene una lista de fechas que cumplen esta condici\'on, medidas en t\'erminos de \textit{d\'ia juliano modificado} o \textit{Modified Julian Date} (MJD). Estos valores, de tipo punto flotante, son ordenados de forma creciente.
\bigskip

\item \textbf{\texttt{select\_fits(dir)}}\\
Selecciona y ordena los elementos de la lista de im\'agenes de formato \textsc{fits} del directorio \texttt{dir} usando la lista de MJDs (guardada en la variable de instancia \texttt{mjd} de la clase) como resultado de \texttt{filter\_science\_images()} escogiendo s\'olo aquellas im\'agenes cuyas fechas correspondan a las fechas indicadas.
\bigskip

\item \textbf{\texttt{select\_npys(dir, ref\_dir, init\_index, n\_pos, rest\_len)}}:\\
Debido a que los archivos de extensi\'on NPY no poseen la variable MJD en su estructura (en los archivos \textsc{fits} encontramos este valor en el header de la imagen) deben de filtrarse de forma diferente. Para este caso el filtrado de este tipo de archivos se lleva a cabo a trav\'es de la revisi\'on de sus nombres, ya que comparten patrones con los nombres de ciertas im\'agenes \textsc{fits}. Por ejemplo, los nombres de las im\'agenes de PSF, de formato NPY, poseen similitud con los nombres de las im\'agenes \textsc{fits} de diferencia; igualmente los archivos \texttt{aflux} de formato NPY poseen parecidos en sus nombres con las im\'agenes cient\'ificas. Esta similitud es medida a trav\'es de un substring diferente para cada tipo de archivo NPY, definido por la posici\'on inicial \texttt{init\_index}, en el nombre del archivo \textsc{fits} y largo \texttt{rest\_len}. \texttt{n\_pos} indica la posici\'on de un car\'acter espec\'ifico `\_ ' en dicho substring para validar esta comparaci\'on.

\end{itemize}
\pagebreak


\section{M\'etodos de la clase \textsc{SourceFinder}}
\label{ap:sourcefinder}
\begin{itemize}
\item \texttt{pixel\_discard()}:\\
M\'etodo en el que se realiza el descarte de p\'ixeles de forma individual, siguiendo los siguientes criterios:
\begin{enumerate}
\item Si el flujo estimado por el filtro de Kalman para un pixel es menor que el umbral dado.
\item Si la velocidad de flujo estimada por el filtro de Kalman es menor que el umbral  de la velocidad de flujo multiplicado por la tasa de saturaci\'on en la velocidad de flujo.
\item Si un pixel de la imagen cient\'ifica es menor a la mediana m\'as cierto delta (en este trabajo, siguiendo la l\'inea de desarrollo de P. Huentelemu \cite{huentelemu}, se consider\'o 5.0) es descartado.
\item Si las varianzas de flujo son mayores a 150.0 (valor estimado por el autor del software original \cite{huentelemu}).
\item Si las varianzas de la tasa de cambio de flujo (o velocidad de flujo) es mayor o igual a 150.0.
\item Si los p\'ixeles no caen en etiquetas de invalidaci\'on dentro de la m\'ascara que ha sido procesada para marcar tambi\'en los p\'ixeles vecinos a los realmente defectuosos.
\item Si los p\'ixeles no han ca\'ido dentro del descarte por superar la mediana estimada a partir de cuatro \'epocas. 
\end{enumerate}
\item \texttt{grouping\_pixels()}:\\
Este m\'etodo trabaja con las etiquetas determinadas con el m\'etodo anterior, en un arreglo de matrices (\texttt{numpy array}) denominado \texttt{pixel\_flags} (variable de instancia). Adem\'as recibe el \'indice de MJD correspondiente a la observaci\'on de tal fecha.
La agrupaci\'on de p\'ixeles se realiza gracias a funciones brindadas por la librer\'ia Mahotas, usando el m\'etodo \texttt{label} para encontrar dominios cerrados en el mapa de p\'ixeles validados.
\bigskip

\item \texttt{filter\_groups(science, flux, var\_flux, state, base\_mask)}:\\
Este m\'etodo recibe la imagen cient\'ifica, el flujo y su varianza, el estado determinado por el filtro de Kalman y la m\'ascara correspondiente a una \'epoca espec\'ifica. 
El filtrado de grupos de p\'ixeles se lleva a cabo bajo las siguientes reglas de descarte: 

\begin{enumerate}
\item Descarte de grupo por contener posible mala resta alrededor (valores negativos).  
\item	Si no hay m\'aximos locales dentro del grupo de p\'ixeles encontrados dentro de la imagen cient\'ifica.
\item	Si no hay m\'aximos locales dentro del grupo de p\'ixeles encontrados en la matriz de flujo (calculado por \texttt{calc\_fluxes}).
\item	Si no hay m\'aximos locales dentro del grupo de p\'ixeles encontrados en la matriz velocidad de flujo.
\item 	Si los valores de los p\'ixeles superan la mediana local en imagen cient\'ifica.
\item	Si el grupo posee alg\'un pixel que doble el valor del flujo o de la imagen cient\'ifica.
\item	Si el centro del grupo se encuentra etiquetado como defectuoso dentro de la m\'ascara.
\item	Si el pixel del centro del grupo se encuentra rechazado al ser superior a la mediana de los p\'ixeles de cuatro observaciones consecutivas.
\item	Si la varianza del flujo del pixel del centro del grupo es mayor al determinado por el umbral.
\end{enumerate} 
\item \texttt{update\_candidates(mjd):}\\
En la estructura \texttt{CandData} se van registrando fechas (MJD) en que se han detectado candidatos previamente o se han detectado por primera vez. Es una estrutura tipo lista en la que se van guardando diccionarios que contienen, cada uno, las coordenadas de un objeto, las \'epocas en las que ha sido detectado y si corresponde o no a una supernova conocida.

\item \texttt{check\_candidates(SN\_index, SN\_pos):}\\
Verifica que dado un \'indice de supernova (\texttt{SN\_index}) y sus coordenadas \texttt{SN\_pos} se ha detectado dentro de los candidatos encontrados. 

\item \texttt{draw\_complying\_pixel\_groups(science, state, state\_cov, base\_mask, dil\_mask, flux, var\_flux, mjd)}:\\
Este m\'etodo es el que llama a \texttt{pixel\_discard} para etiquetar p\'ixeles para el descarte y no ser considerados en el paso de agrupamiento al llamar a \texttt{grouping\_pixels}. Luego se invoca el m\'etodo \texttt{filter\_groups} para hacer el descarte a nivel grupal y obtener candidatos. La \'ultima llamada es para el m\'etodo \texttt{update\_candidates} para actualizar lista de candidatos encontrados en la variable de instancia \texttt{CandData}.
\bigskip

Como argumentos recibe todos los elementos necesarios para ejecutar los m\'etodos que llama.
%\texttt{save\_data} que se encuentra en clase \textsc{DataContent}. 

\end{itemize}
\pagebreak

\section{M\'etodos en \texttt{utils}}
\label{ap:utils}
\begin{itemize}
\item \textbf{\texttt{naylor\_photometry(invvar, diff, psf)}:}\\
Calcula el producto del flujo por su varianza. Retorna el producto y la varianza. Para esto obtiene el flujo a partir de la imagen PSF entregada (\texttt{psf}) y del producto entre la imagen diferencia y la de varianza inversa (\texttt{diff} y \texttt{invvar} respectivamente)\cite{naylor}.
\bigskip


\item \textbf{\texttt{calc\_fluxes(diff, psf, invvar, aflux)}:}\\
Calcula el flujo y su varianza gestionando la entrada y la salida de \texttt{naylor\_photometry(invvar, diff, psf)}. Los valores NaN son transformados a valor constante 0.001.
\bigskip

\item \texttt{subsampled\_median(image, image\_size, sampling):}\\
Extrae un n\'umero de \texttt{sampling} muestras de la imagen \texttt{image} con las cuales se estima su mediana. 
\bigskip

\item \texttt{cholesky(P):}\\
Calcula la descomposici\'on de Cholesky de una estructura matricial de profundidad 3, en particular para operar sobre covarianzas.
\bigskip

\item \texttt{mask\_and\_dilation(mask\_path):}\\
Aplica una funci\'on sobre la m\'ascara (ubicada en la ruta \texttt{mask\_path}) para amplificar la zona de invalidez en torno a los p\'ixeles que han sido marcados como defectuosos.
\bigskip

\end{itemize} 
\pagebreak

\section{M\'etodos en \texttt{unscented\_utils}}
\label{ap:unsutils}
Se desarrollaron diferentes funciones auxiliares para apoyar el c\'alculo de las matrices:
\begin{itemize}
\item \texttt{sigma\_points(mean\_, cov\_, lambda\_, N):}\\
Funci\'on con la cual se calculan los puntos sigma a partir de la media de las variables de estado, la covarianza, el valor de $\lambda$ y el n\'umero de variables de estado, N. Utiliza para esta finalidad, la descomposici\'on de Cholesky.
\item \texttt{unscent\_weights(kappa, alpha, beta, N):}\\
M\'etodo con el que se calculan los pesos a partir de los valores de $\kappa$, $\alpha$, $\beta$ y N (n\'umero de variables de estado). 
\item \texttt{perform(func, *args):}\\
Funci\'on auxiliar con la cual se recibe un puntero a otra funci\'on vectorial (destinada a ser aplicada sobre un conjunto de puntos sigma) y un n\'umero arbitrario de argumentos, dependiendo de la necesidad de la misma funci\'on.
\item \texttt{propagate\_func(func, Wm, Wc, Xs, *args, N):}\\
Funci\'on con la que se propaga la funci\'on \texttt{func} sobre el conjunto de puntos sigma \texttt{Xs} usando los pesos de media y covarianza \texttt{Wm} y \texttt{Wc}, respectivamente, adem\'as del n\'umero de variables de estado, N. Adem\'as, recibe el argumento \texttt{args} que corresponde a una tupla de entradas propias de la funci\'on \texttt{func}. 
\end{itemize}
\pagebreak

\section{Detecci\'on usando pipeline original}
\begin{table}[h!]
\small
\centering
\caption{Resultados de \'epocas de detecci\'on en t\'erminos de MJD de las 93 supernovas del conjunto de 2015 de HiTS, usando los filtros implementados originalmente (b\'asico y de correntrop\'ia m\'axima).}
\begin{tabular}{|l|r|r|}
\hline
\textbf{\'Ind.} & \textbf{B\'asico} & \textbf{MCC}   \\
\hline
1&57072,19 & 57072,19 \\
2&-             & -             \\
3&57075,15 & 57075,15 \\
4&57075,21 & 57075,21 \\
5&57072,24 & 57072,24 \\
6&-             & -             \\
7&-             & -             \\
8&57075,10 & 57075,10 \\
9&57075,22 & 57075,22 \\
10&57077,11 & 57077,11 \\
11&57075,20 & 57075,20 \\
12&57072,21 & 57072,21 \\
13&57077,15 & 57077,15 \\
14&-             & -             \\
15&57077,11 & 57077,11\\
16&57077,09 & 57077,09\\
17&57077,11 & 57077,11 \\
18&57090,23 & 57090,23 \\
19&-             & -             \\
20&57077,12   & 57077,12  \\
21&-             & -             \\
22&57077,12 & 57077,12 \\
23&-             & -             \\
24&57077,08  & 57077,08   \\
25&57075,25 & 57075,25 \\
26&-             & -             \\
27&57077,17 & 57077,17 \\
28&-             & -             \\
29&-             & -             \\
30&57072,35 & 57072,35 \\
31&-             & -             \\
32&57075,21 & 57075,21 \\
33&-             & -             \\
34&-             & -             \\
35&-             & -             \\
36&57080,11 & 57080,11 \\
37&57075,21 & 57075,21 \\
38&-             & -             \\
39&-             & -             \\
40&-             & -             \\
41&-             & -             \\
42&-             & -             \\
43&57090,22 & 57090,22\\
44&-             & -             \\
45&-             & -             \\
46&-             & -             \\
47&57075,20 & 57075,20 \\\hline
\end{tabular}
\quad
\begin{tabular}{|l|r|r|}
\hline
\textbf{\'Ind.} & \textbf{B\'asico} & \textbf{MCC} \\
\hline
48&57080,17 & 57080,17 \\
49&57090,24 & 57090,24 \\
50&57075,21 & 57075,21 \\
51&-             & -             \\
52&-             & -             \\
53&-             & -             \\
54&-             & -             \\
55&-             & -             \\
56&57075,11  & 57075,11  \\
57&57095,20 & 57095,20 \\
58&-             & -             \\
59&57095,20 & 57095,20\\
60&-             & -             \\
61&-             & -             \\
62&57095,16 & 57095,16\\
63&-             & -             \\
64&-             & -             \\
65&-             & -             \\
66&-             & -             \\
67&-             & -             \\
68&-             & -             \\
69&-             & -             \\
70&-             & -             \\
71&-             & -             \\
72&-             & -             \\
73&-             & -             \\
74&-             & -             \\
75&-             & -             \\
76&-             & -             \\
77&-             & -             \\
78&-             & -             \\
79&-             & -             \\
80&-             & -             \\
81&-             & -             \\
82&-             & -             \\
83&-             & -             \\
84&-             & -             \\
85&-             & -             \\
86&57080,20 & 57080,20 \\
87&-             & -             \\
88&-             & -             \\
89&-             & -             \\
90&-             & -             \\
91&NaN           & NaN           \\
92&NaN           & NaN           \\
93&NaN           & NaN          \\\hline
\end{tabular}
\label{ap:tab1}
\end{table}
\pagebreak


\section{Detecci\'on usando pipeline refactorizada}
\label{ap:pip_ref}
\begin{table}[h!]
\small
\centering
\caption{Resultados de \'epocas de detecci\'on en t\'erminos de MJD de las 93 supernovas del conjunto de 2015 de HiTS, usando los filtros refactorizados (b\'asico y de correntrop\'ia m\'axima).}
\begin{tabular}{|l|r|r|}
\hline
\textbf{\'Ind.} & \textbf{B\'asico} & \textbf{MCC} \\\hline
1& 57072,19   & 57072,19   \\
2& -              & -              \\
3& 57075,15    & 57075,15     \\
4& -   & -   \\
5& 57072,24   & 57072,24   \\
6& 57077,09   & 57077,09   \\
7& -              & -              \\
8& 57075,10   & 57075,10   \\
9& 57075,22   & 57075,22   \\
10& 57077,11   & 57077,11   \\
11& 57075,20    & 57075,20    \\
12& 57072,14   & 57072,14   \\
13& 57077,15   & 57077,15   \\
14& -    & -    \\
15& 57077,11   & 57077,11   \\
16& 57077,09   & 57077,09   \\
17& 57077,11   & 57077,11   \\
18& -   & -   \\
19& -              & -              \\
20& 57077,18   & 57077,18   \\
21& 57077,12   & 57077,18   \\
22& 57077,12  & 57077,12   \\
23& -              & -              \\
24& 57077,08    & 57077,08             \\
25& 57077,09   & 57077,09   \\
26& -   & -   \\
27& 57077,17   & 57077,17   \\
28& -              & -              \\
29& -              & -              \\
30& 57072,21   & 57072,21   \\
31& -              & -              \\
32& 57075,21   & 57075,21   \\
33& -              & -              \\
34& -              & -              \\
35& 57078,20   & 57078,20   \\
36& 57080,11   & 57080,11   \\
37& -              & -              \\
38& - & - \\
39& -              & -              \\
40& -              & -              \\
41& 57075,08     & 57075,08              \\
42& -              & -              \\
43& 57090,22   & 57090,22   \\
44& -              & -              \\
45& -              & -              \\
46& -              & -              \\
47& 57075,20   & 57075,20   \\\hline
\end{tabular}
\quad
\begin{tabular}{|l|r|r|}
\hline
\textbf{\'Ind.} & \textbf{B\'asico} & \textbf{MCC}\\\hline
48& 57080,11   & 57080,11   \\
49& 57090,24   & 57090,24   \\
50& 57077,18   & 57077,18   \\
51& 57077,13   & 57077,13  \\
52& -              & -              \\
53& -              & -              \\
54& 57095,17              & 57095,17              \\
55& -              & -              \\
56& 57075,11   & 57075,11   \\
57& 57095,20   & 57095,20   \\
58& -              & -              \\
59& -              & -              \\
60& 57095,20              & 57095,20             \\
61& -              & -              \\
62& -   & -   \\
63& -              & -              \\
64& -              & -              \\
65& -              & -              \\
66& 57095,15   & 57095,15   \\
67& -              & -              \\
68& -              & -              \\
69& -              & -              \\
70& -              & -              \\
71& -              & -              \\
72& -              & -              \\
73& -              & -              \\
74& -              & -              \\
75& -              & -              \\
76& -              & -              \\
77& -              & -              \\
78& -              & -              \\
79& -              & -              \\
80& -              & -              \\
81& -              & -              \\
82& -              & -              \\
83& -              & -              \\
84& -              & -              \\
85& -              & -              \\
86& 57090,26   & 57090,26   \\
87& -              & -              \\
88& -              & -              \\
89& -              & -              \\
90& -              & -              \\
91& NaN            & NaN            \\
92& NaN            & NaN            \\
93& NaN            & NaN           \\\hline
\end{tabular}
\label{ap:tab2}
\end{table}
\pagebreak

\section{Detecci\'on usando filtro unscented}
\begin{table}[h!]
\small
\centering
\caption{Resultados de \'epocas de detecci\'on en t\'erminos de MJD de las 93 supernovas del conjunto de 2015 de HiTS, usando el filtro unscented para una funci\'on $f(\Delta t) = \Delta t^{1.5}$ y $f(\Delta t) = \Delta t^{2}$.}

\begin{tabular}{|l|r|r|}
\hline
\textbf{\'Ind.} & \textbf{n=1.5}  & \textbf{n=2.0}  \\\hline
1     & -        & 57072,26 \\
2     & -        & -        \\
3     & -        & -        \\
4     & -        & -        \\
5     & 57072,24 & -        \\
6     & -        & -        \\
7     & -        & -        \\
8     & -        & 57077,09 \\
9     & -        & -        \\
10    & -        & -        \\
11    & -        & -        \\
12    & -        & -        \\
13    & -        & -        \\
14    & -        & -        \\
15    & -        & -        \\
16    & -        & -        \\
17    & -        & -        \\
18    & -        & -        \\
19    & -        & -        \\
20    & 57090,24 & 57090,24 \\
21    & -        & -        \\
22    & -        & -        \\
23    & -        & -        \\
24    & -        & -        \\
25    & -        & -        \\
26    & -        & -        \\
27    & -        & -        \\
28    & -        & -        \\
29    & -        & -        \\
30    & -        & -        \\
31    & -        & -        \\
32    & -        & -        \\
33    & -        & -        \\
34    & -        & -        \\
35    & -        & -        \\
36    & -        & -        \\
37    & -        & -        \\
38    & -        & -        \\
39    & -        & -        \\
40    & -        & -        \\
41    & -        & -        \\
42    & -        & -        \\
43    & -        & -        \\
44    & -        & -        \\
45    & -        & -        \\
46    & -        & -        \\
47    & -        & -        \\\hline
\end{tabular}
\quad
\begin{tabular}{|l|r|r|}
\hline
\textbf{\'Ind.} & \textbf{n=1.5} & \textbf{n=2.0}\\\hline
48    & -        & -        \\
49    & -        & -        \\
50    & -        & -        \\
51    & -        & -        \\
52    & -        & -        \\
53    & -        & -        \\
54    & -        & -        \\
55    & -        & -        \\
56    & -        & -        \\
57    & -        & -        \\
58    & -        & -        \\
59    & -        & -        \\
60    & -        & -        \\
61    & -        & -        \\
62    & -        & -        \\
63    & -        & -        \\
64    & -        & -        \\
65    & -        & -        \\
66    & -        & -        \\
67    & -        & -        \\
68    & -        & -        \\
69    & -        & -        \\
70    & -        & -        \\
71    & -        & -        \\
72    & -        & -        \\
73    & -        & -        \\
74    & -        & -        \\
75    & -        & -        \\
76    & -        & -        \\
77    & -        & -        \\
78    & -        & -        \\
79    & -        & -        \\
80    & -        & -        \\
81    & -        & -        \\
82    & -        & -        \\
83    & -        & -        \\
84    & -        & -        \\
85    & -        & -        \\
86    & -        & -        \\
87    & -        & -        \\
88    & -        & -        \\
89    & -        & -        \\
90    & -        & -        \\
91    & NaN        & NaN        \\
92    & NaN        & NaN     \\
93    & NaN        & NaN      \\\hline
\end{tabular}
\label{ap:tab3}
\end{table}
\pagebreak
\begin{landscape}
\section{Diagrama de clases de programa original}
\begin{figure}[h!]
\centering
\includegraphics[scale=.25]{/home/paloma/Documents/Memoria/Code/sif2/sif/sif.png}
\caption{Diagrama de clases del programa original. Se listan cada uno de m\'etodos y variables de cada clase.}
\label{fig:sif_class}
\end{figure}
\end{landscape}

\pagebreak
\begin{landscape}
\section{Diagrama de clases del programa refactorizado}
\begin{figure}[h!]
\centering
\includegraphics[scale=.5]{/home/paloma/Documents/Memoria/Code/sif2/sif2_class}
\caption{Diagrama de clases del programa refactorizado. Se muestra la relaci\'on entre las clases resultantes del proceso de refactoring. Se ha obviado la familia de filtros de Kalman por razones de dimensionalidad de imagen.}
\label{fig:class_sif2}
\end{figure}
\end{landscape}
\end{appendix}



% Bibliografía
\bibliographystyle{abbrv}
\bibliography{./bibliography/tesis}

\end{document}