\chapter{Introducción}
\label{ch:introduction}

\section{Motivaci\'on}
La astronomía es uno de los campos científicos que más se ha visto afectado por el rápido crecimiento en la generación de datos debido al fuerte desarrollo de nuevas tecnologías de la información y  de nuevos instrumentos destinados a la observación. Este crecimiento ha gatillado un aumento importante en la demanda de una nueva generaci\'on de m\'etodos que puedan procesar esta oleada de informaci\'on o big data astron\'omico.
\bigskip

Ejemplos de proyectos que actualmente producen una gran cantidad de datos a trav\'es de telescopios en diferentes partes del mundo son: el Panoramic Survey Telescope and Rapid Response System (Pan-STARRS), el Visible and Infrarred Survey Telescope (VISTA), el VLT Survey Telescope (VST), el Dark Energy Camera Legacy Survey (DECaLS) y el Hyper Suprime-Cam Subaru Strategic Program (HSC SSP). Estos surveys est\'an caracterizados por un amplio \textit{\'etendue}  que consiste en el producto entre el \'area del espejo de un telescopio y su \'angulo s\'olido proyectado en el cielo.
\bigskip

En el futuro, telescopios como el Large Synoptic Survey Telescope (LSST)~\cite{lsst} (que entrará en funcionamiento a mediados del 2022) continuar\'an revolucionando la era del big data en astronom\'ia con etendues y c\'amaras CCD mucho m\'as grandes de lo que hasta el d\'ia de hoy se utiliza. En particular se espera que el LSST produzca un n\'umero de alertas transientes (avisos de objetos encontrados y su consecuente seguimiento) del orden del mill\'on cada noche. La capacidad de detectar nuevos objetos de inter\'es depender\'a de la calidad de los datos y de los algoritmos de tiempo real  destinados a generar las alertas mencionadas. 
\bigskip

Al d\'ia de hoy se han elaborado sondeos como el High Cadence Transient Survey (HiTS)~\cite{hits} cuya finalidad ha sido la b\'usqueda de fen\'omenos transientes r\'apidos con escalas de tiempo que van desde las horas a d\'ias, utilizando secuencias de observaciones de la c\'amara DECam del telescopio Blanco (en Cerro Tololo) para la detecci\'on y posterior reporte de objetos candidatos a supernova. En este trabajo proponen un m\'etodo de detecci\'on de potenciales candidatos a supernova a trav\'es de la discriminaci\'on de p\'ixeles que involucren un incremento trasiente de intensidad.
\bigskip

Esta discriminaci\'on comienza con la determinaci\'on del flujo a trav\'es de la intensidad de los p\'ixeles que puedan corresponder a una estrella y a la variaci\'on de cada uno de ellos usando m\'etodos iterativos de filtrado en secuencias de im\'agenes de largo arbitrario. Los filtros pensados para HiTS corresponden a miembros de una familia de filtros conocidos como \textit{filtros de Kalman}. En particular se hace uso de los filtros de Kalman b\'asico \cite{kalman} y de correntrop\'ia m\'axima \cite{chen}.
\bigskip


El reciente trabajo desarrollado por Pablo Huentelemu~\cite{huentelemu} (M.Sc.) guiado por Pablo Est\'evez  (PhD) y Francisco Förster (PhD), propone el uso de estos filtros en el reconocimiento de supernovas j\'ovenes (o en su fase de crecimiento de luminosidad) de tipo II, por lo que se plantea la posibilidad de dise\~nar alg\'un otro criterio de filtrado y de estudiar la variaci\'on de los resultados.
\bigskip

%En el presente trabajo se pretende desarrollar un nuevo m\'odulo de filtro de Kalman que sea robusto a sistemas no-lineales para la detecci\'on de potenciales candidatos a supernova, para posteriormente ser embebido en un programa que pueda funcionar en l\'inea generando alarmas de detecci\'on. Por otro lado se busca implementar un software que contenga los m\'etodos de filtrados originales (versiones cl\'asica y de m\'axima correntrop\'ia del filtro de Kalman) a trav\'es de un refactoring del programa  base. Paralelamente, se desea encontrar rangos apropiados para los umbrales requeridos para el filtrado.


\section{Objetivos}
El objetivo de esta tesis comprende la reestructuraci\'on de un programa destinado al an\'alisis fotom\'etrico de datos astron\'omicos con \'el cual se busca implementar una base de sistema de alertas que de aviso del estado temprano de una posible supernova de tipo II. Es decir, que notifique el momento en que la luminosidad de una estrella comience a crecer debido a este tipo de explosi\'on. 
\bigskip

El llevar a cabo este refactorizaci\'on requiere la familiarizaci\'on con una familia de m\'etodos conocido como Filtros de Kalman que hacen uso de las mediciones obtenidas sobre un sistema din\'amico empleando la incertidumbre de las observaciones y del sistema en la estimaci\'on del estado de este. Esto, con la finalidad de poder desarrollar una familia de m\'etodos en t\'erminos de programaci\'on orientada a objetos empleando el patr\'on de dise\~no Strategy.
\bigskip

La refactorizaci\'on de los m\'etodos de filtrado existentes (b\'asico y de m\'axima correntrop\'ia ~\cite{huentelemu}) va acompa\~nada del desarrollo de una nueva variante que permita el uso de un modelo no lineal y control sobre \'el mismo (es decir, que el usuario pueda decidir qu\'e modelo no-lineal emplear para sus estimaciones).
\bigskip

Por otro lado, se requiere un breve proceso de ingenier\'ia sobre la manipulaci\'on de los archivos requisito del programa original, debido a que existen problemas de \textit{hard-coding} respecto de la ubicaci\'on de estos, as\'i como ciertos reparos durante el proceso de selecci\'on de los mismos. Adem\'as se necesita reformular la configuraci\'on de par\'ametros funcionales de entrada (propios de la ejecuci\'on del programa, como los umbrales) para flexibilizar la configuraci\'on de estos \'ultimos. 
\bigskip

Por otra parte, se desea implementar una nueva alternativa en la obtenci\'on de las gr\'aficas de resultados en las que sea posible observar la medici\'on y error de la entrop\'ia al obtener las curvas de estado (espacio de fase) de flujo y velocidad de flujo. 
\bigskip

Finalmente se busca estudiar los resultados obtenidos con esta nueva versi\'on del programa con cada uno de sus filtros sobre un conjunto de datos brindado por HiTS (del a\~no 2015), a la par que el desempe\~no de este. Por \'ultimo se busca establecer un contraste con la versi\'on original.     


\section{Organizaci\'on de la tesis}

Este documento sigue los siguientes cap\'itulos: en el Cap\'itulo \ref{ch:background} se describe en qu\'e consiste una supernova, el proyecto HiTS  as\'i como el background matem\'atico de los m\'etodos de filtrado usados en el trabajo original (filtros de Kalman b\'asico y de m\'axima correntrop\'ia) y el filtro a implementar (filtro de Kalman unscented).%la relevancia de la linealidad de un sistema (y su impacto en la estad\'istica del mismo en la aplicaci\'on de un m\'etodo de filtrado como el filtro de Kalman)%
\bigskip

%El siguiente cap\'itulo, \ref{ch:linear} est\'a destinado para explicitar la relevancia de la linealidad de un modelo y que ocurre en casos de que este no sea, al aplicar un filtro de Kalman.
%\bigskip

En el Cap\'itulo \ref{ch:prev_work}, se estudian tanto el desempe\~no como los resultados obtenidos con el c\'odigo original del programa sobre el cual se trabajar\'a, estableci\'endose una breve discusi\'on sobre estos resultados. 
\bigskip

Posteriormente, en el Cap\'itulo \ref{ch:refactoring} se describe el refactoring del programa  del c\'odigo original: se enlistan nuevos m\'etodos para el manejo de archivos y cambios realizados en la implementaci\'on de los m\'etodos iniciales. 
\bigskip

En el Cap\'itulo \ref{ch:news} se exponen los pasos del desarrollo del nuevo filtro adem\'as de la nueva funcionalidad relacionada con la carga de resultados previamente guardados en pos de hacer este programa un proceso on-line.
\bigskip

Posteriormente, en el cap\'itulo \ref{ch:resultados}, se exponen los resultados obtenidos para el desempe\~no de la nueva versi\'on del programa (para los dos filtros originales) adem\'as de la cantidad de supernovas redescubiertas y el periodo en que se detectaron. El cap\'itulo concluye con el an\'alisis de los estos resultados y una comparaci\'on con aquellos obtenidos en el Cap\'itulo \ref{ch:prev_work}.
\bigskip

Finalmente las conclusiones, contenidas en el cap\'itulo \ref{ch:conclusion}, resumen el aprendizaje obtenido durante este trabajo as\'i como los resultados relevantes logrados durante las pruebas realizadas. Adem\'as se plantean nuevas l\'ineas de trabajo futuro.


%\fillup{1}

%\todo{todo 1} asdfasfsdfsadafs \missingref{me faltaría una ref!}.
%\todo[inline]{todo 2}