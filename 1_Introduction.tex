\chapter{Introducción}
\label{ch:introduction}

\section{Motivaci\'on}
La astronomía es uno de los campos científicos que más se ha visto afectado por el rápido crecimiento en la generación de datos debido al fuerte desarrollo de nuevas tecnologías de la información y  de nuevos instrumentos destinados a la observación. Este crecimiento ha gatillado un aumento importante en la demanda de una nueva generaci\'on de m\'etodos que puedan procesar esta oleada de informaci\'on o big data astron\'omico.
\bigskip

Ejemplos de proyectos que actualmente producen una gran cantidad de datos a trav\'es de telescopios en diferentes partes del mundo son: el Panoramic Survey Telescope and Rapid Response System (Pan-STARRS)\cite{pan}, el Visible and Infrarred Survey Telescope (VISTA)\cite{vista}, el VLT Survey Telescope (VST)\cite{vst}, el Dark Energy Camera Legacy Survey (DECaLS)\cite{decals} y el Hyper Suprime-Cam Subaru Strategic Program (HSC SSP)\cite{ssp}. Estos surveys est\'an caracterizados por un amplio \textit{\'etendue} definido como el producto entre el \'area del espejo de un telescopio y su \'angulo s\'olido proyectado en el cielo.
\bigskip

En el futuro, telescopios como el Large Synoptic Survey Telescope (LSST)~\cite{lsst} (que entrará en funcionamiento a mediados del 2022) continuar\'an revolucionando la era del big data en astronom\'ia con \'etendues y c\'amaras CCD mucho m\'as grandes de lo que se utiliza hasta el d\'ia de hoy. En particular se espera que el LSST produzca un n\'umero de alertas de fen\'omenos transitorios (avisos de objetos que cambian en el tiempo o espacio) del orden de 10 millones cada noche. La capacidad de detectar nuevos objetos de inter\'es depender\'a de la calidad de los datos y de los algoritmos de tiempo real  destinados a generar las alertas mencionadas. 
\bigskip

Al d\'ia de hoy se han elaborado sondeos como el High Cadence Transient Survey (HiTS)~\cite{hits} cuya finalidad ha sido la b\'usqueda de fen\'omenos transitorios r\'apidos con escalas de tiempo que van desde las horas a d\'ias, utilizando secuencias de observaciones de la c\'amara DECam del telescopio Blanco (en Cerro Tololo) para la detecci\'on y posterior reporte de objetos candidatos a supernova. En este trabajo se propone un m\'etodo de detecci\'on de potenciales candidatos a supernova a trav\'es de la discriminaci\'on de p\'ixeles que involucren un incremento en la intensidad. Esta discriminaci\'on comienza con la determinaci\'on del flujo a trav\'es de la intensidad de los p\'ixeles que puedan corresponder a una estrella y a la variaci\'on de cada uno de ellos usando m\'etodos iterativos de filtrado en secuencias de im\'agenes de largo arbitrario. Los filtros desarrollados para HiTS corresponden a miembros de una familia de filtros conocidos como \textit{filtros de Kalman}.% En particular se hace uso de los filtros de Kalman b\'asico \cite{kalman} y de correntrop\'ia m\'axima \cite{chen}.
\bigskip


El trabajo desarrollado por Pablo Huentelemu~\cite{huentelemu} (M.Sc.), propone el uso de los filtros cl\'asico (o b\'asico) \cite{kalman} y de m\'axima correntrop\'ia \cite{chen} en el reconocimiento de supernovas j\'ovenes (o en su fase de crecimiento de luminosidad) de tipo II, pero que a pesar de acertar en algunas detecciones (es decir, supernovas confirmadas), se busca mejorar su resultado. Por lo que se plantea la posibilidad de dise\~nar alg\'un otro criterio de filtrado y de estudiar la variaci\'on de los resultados.
\bigskip


\section{Objetivos}
\subsection{Objetivo general}
El objetivo de esta tesis comprende la reestructuraci\'on y extensi\'on de un programa existente destinado al an\'alisis fotom\'etrico de datos astron\'omicos con \'el cual se busca implementar un modelo de sistema de alertas que d\'e aviso del estado temprano de un fen\'omeno astron\'omico transitorio. 
\bigskip

\subsection{Objetivos espec\'ificos}

\begin{itemize}


\item Mejorar proceso existente sobre la manipulaci\'on de los archivos requisito del programa original, debido a que actualmente existen problemas de \textit{hard-coding} respecto de la ubicaci\'on de estos, as\'i como ciertos reparos durante el proceso de selecci\'on de los mismos. Adem\'as se necesita reformular la configuraci\'on de par\'ametros funcionales de entrada (propios de la ejecuci\'on del programa, como los umbrales) para flexibilizar su entrada. 
\bigskip

\item Generar una familia de m\'etodos en t\'erminos de programaci\'on orientada a objetos empleando el patr\'on de dise\~no \textit{Strategy} que implemente la familia de m\'etodos conocida como Filtros de Kalman.
\bigskip

\item Agregar una nueva variante de filtro que permita el uso de un modelo no lineal.
\bigskip

\item Implementar gr\'aficas de resultados en las que sea posible observar la curva generada por el flujo y la velocidad de flujo estimados por los filtros, indicando la entrop\'ia aproximada de la curva en el espacio de estados y visualizar la evoluci\'on de las mediciones y estimaciones tanto en secuencias de im\'agenes estampillas (p\'ixeles) como en gr\'aficas.    
\bigskip

\item Estudiar los resultados obtenidos con esta nueva versi\'on del programa con cada uno de sus filtros sobre un conjunto de datos brindado por HiTS (del a\~no 2015), a la par que el desempe\~no (medido en t\'erminos de uso de memoria principal y tiempo de ejecuci\'on) de \'este. Por \'ultimo se busca establecer un contraste con la versi\'on original.     

\end{itemize}
\section{Organizaci\'on de la tesis}

Este documento contiene los siguientes cap\'itulos: en el Cap\'itulo \ref{ch:background} se describe en qu\'e consiste una supernova, el proyecto HiTS  as\'i como el background matem\'atico de los m\'etodos de filtrado usados en el trabajo original (filtros de Kalman b\'asico y de m\'axima correntrop\'ia) y el filtro a implementar (filtro de Kalman unscented).%la relevancia de la linealidad de un sistema (y su impacto en la estad\'istica del mismo en la aplicaci\'on de un m\'etodo de filtrado como el filtro de Kalman)%
\bigskip

En el Cap\'itulo \ref{ch:prev_work}, se describen aspectos generales del programa original y se estudian los resultados de desempe\~no obtenidos con \'el, estableci\'endose una breve discusi\'on. 
\bigskip

Posteriormente, en el Cap\'itulo \ref{ch:refactoring} se describe el refactoring del programa  del c\'odigo original y se detalla la construcci\'on de la familia de m\'etodos de filtros de Kalman usando el patr\'on Strategy y del redise\~no de las funciones de visualizaci\'on, incluyendo la generaci\'on de la curva de estados.
\bigskip

Luego, en el Cap\'itulo \ref{ch:news} se detalla la implementaci\'on del nuevo filtro de aproximaci\'on no lineal.
\bigskip

A continuaci\'on, en el Cap\'itulo \ref{ch:resultados}, se exponen los resultados obtenidos para el desempe\~no de la nueva versi\'on del programa (para los dos filtros originales) incluyendo la cantidad de supernovas redescubiertas y el per\'iodo en que se detectaron. El cap\'itulo concluye con el an\'alisis de los resultados y una comparaci\'on con aquellos obtenidos en el Cap\'itulo \ref{ch:prev_work}.
\bigskip

Finalmente en el Cap\'itulo \ref{ch:conclusion} se presentan las conclusiones de este trabajo y se propone el trabajo a futuro a realizar.

%\fillup{1}

%\todo{todo 1} asdfasfsdfsadafs \missingref{me faltaría una ref!}.
%\todo[inline]{todo 2}